%-------------------------------------------------------------------------------------------------------
%-------------------------------------------------------------------------------------------------------
\chapter{Message Formats}

Before diving into the actual design of the software and hardware components, let us first outline the message data formats as they flow on the layout control bus. It is the foundation of the layout control system, so let's have a first brief look at all the messages defined. This chapter will provide the overview on the available messages and give a short introduction to what they do. Later chapters build on it and explain how the messages are used for designing LCS node functions. The layout control system messages can be grouped into several categories:

\begin{itemize}
    \item General management
    \item Node and Port management
    \item Event management
    \item DCC Track management
    \item DCC Locomotive Decoder management
    \item DCC Accessory Decoder management
    \item RailCom DCC Packet management
    \item Raw DCC Packet management
\end{itemize}

All nodes communicate via the layout control bus by broadcasting messages. Every node can send a message, and every node receives the message broadcasted. There is no central master. The current implementation is using the CAN bus, which ensures by definition that a message is correctly transmitted. However, it does not guarantee that the receiver actually processed the message. For critical messages, a request-reply scheme is implemented on top. Also, to address possible bus congestion, a priority scheme for messages is implemented to ensure that each message has a chance for being transmitted.

\section{LCS Message Format}

A message is a data packet of up to 8 bytes. The first byte represents the operation code. It encodes the length of the entire packet and opcode number. The first 3 bits represent the length of the message, the remaining 5 bits represent the opCode. For a given message length, there are 32 possible opcode numbers. The last opcode number in each group, 0x1F, is reserved for possible extensions of the opcode number range. The remaining bytes are the data bytes, and there can be zero to seven bytes. The message format is independent of the underlying transport method. If the bus technology were replaced, the payload would still be the same. For example, an Ethernet gateway could send those messages via the UDP protocol. The messages often contain 16-bit values. They are stored in two bytes, the most significant byte first and labeled ``xxx-H'' in the message descriptions to come. The message format shown in the tables of this chapter just presents the opCode mnemonic. The actual value can be found in the core library include file.

\section{General Management}

The general management message group contains commands for dealing with the layout system itself. The reset command \texttt{(RESET)} directs all hardware modules, a node, or a port on a node to perform a reset. The entire bus itself can be turned on and off \texttt{(BUS-ON, BUS-OFF)}, enabling or suppressing the message flow. Once the bus is off, all nodes wait for the bus to be turned on again. Finally, there are messages for pinging a node \texttt{(PING)} and request acknowledgement \texttt{(ACK/ERR)}.

\begin{table}[h!]
    \begin{center}
        \caption{General Management}
        \begin{tabular}{|l|l|l|l|l|l|l|l|}
            \toprule
            \textbf{Opcode} & \textbf{Data1} & \textbf{Data2} & \textbf{Data3} & \textbf{Data4} & \textbf{Data5} & \textbf{Data6} & \textbf{Data7} \\
            \midrule
            RESET & npId-H & npId-L & flags & & & & \\
            BUS-ON & & & & & & & \\
            BUS-OFF & & & & & & & \\
            SYS-TIME & arg1 & arg2 & arg3 & arg4 & & & \\
            LCS-INFO & arg1 & arg2 & arg3 & arg4 & & & \\
            PING & npId-H & npId-L & & & & & \\
            ACK & npId-H & npId-L & & & & & \\
            ERR & npId-H & npId-L & code & arg1 & arg2 & & \\
            \bottomrule
        \end{tabular}
    \end{center}
\end{table}

\subsection*{Additional Notes}
\begin{itemize}
    \item Do we need a message for a central system time concept?
    \item Do we need a message for a message that describes the global LCS capabilities?
    \item Do we need an emergency stop message that every node can emit?
\end{itemize}

\section{Node and Port Management}

When a hardware module is powered on, the first task is to establish the node Id in order to broadcast and receive messages. The \texttt{(REQ-NID)} and \texttt{(REP-ID)} messages are the messages used to implement the protocol for establishing the nodeId. More on this in the chapter on message protocols. A virgin node has the hardware module-specific node type and a node Id of \texttt{NIL} also be set directly through the \texttt{(SET-NID)} command. This is typically done by a configuration tool.

\begin{table}[ht!]
    \begin{center}
        \caption{Node and Port Management}
        \begin{tabular}{|l|l|l|l|l|l|l|l|}
            \toprule
            \textbf{Opcode} & \textbf{Data1} & \textbf{Data2} & \textbf{Data3} & \textbf{Data4} & \textbf{Data5} & \textbf{Data6} & \textbf{Data7} \\
            \midrule
            REQ-NID & nId-H & nId-L & nUID-4 & nUID-3 & nUID-2 & nUID-1 & flags \\
            REP-NID & nId-H & nId-L & nUID-4 & nUID-3 & nUID-2 & nUID-1 & flags \\
            SET-NID & nId-H & nId-L & nUID-4 & nUID-3 & nUID-2 & nUID-1 & flags \\
            NCOL    & nId-H & nId-L & nUID-4 & nUID-3 & nUID-2 & nUID-1 & \\
            \bottomrule
        \end{tabular}
    \end{center}
\end{table}


All nodes monitor the message flow to detect a potential node collision. This could be for example the case when a node from one layout is installed in another layout. When a node detects a collision, it will broadcast the \texttt{(NCOL)} message and enter a halt state. Manual interaction is required. A node can be restarted with the \texttt{(RES-NODE)} command, given that it still reacts to messages on the bus. All ports on the node will also be initialized. In addition a specific port on a node can be initialized. The hardware module replies with an \texttt{(ACK)} message for a successful node Id and completes the node Id allocation process. As the messages hows, node and port ID are combined. LCS can accommodate up to 4095 nodes, each of which can host up to 15 ports. A Node ID 0 is the NIL node. Depending on the context, a port Id of zero refers all ports on the node or just the node itself.

The query node \texttt{(NODE-GET)} and node reply messages \texttt{(NODE-REP)} are available to obtain attribute data from the node or port. The \texttt{(NODE-SET)} allows to set attributes for a node or port for the targeted node. Items are numbers assigned to a data location or an activity. There are reserved items such as getting the number of ports, or setting an LED. In addition, the firmware programmer can also define items with node specific meaning. The firmware programmer defined items are accessible via the \texttt{(NODE-REQ)} and \texttt{(NODE-REP)} messages.

\begin{table}[ht!]
    \begin{center}
        \caption{Node and Port Management}
        \begin{tabular}{|l|l|l|l|l|l|l|l|}
            \toprule
            \textbf{Opcode} & \textbf{Data1} & \textbf{Data2} & \textbf{Data3} & \textbf{Data4} & \textbf{Data5} & \textbf{Data6} & \textbf{Data7} \\
            \midrule
            NODE-GET & npId-H & npId-L & item & arg1-H & arg1-L & arg2-H & arg2-L \\
            NODE-PUT & npId-H & npId-L & item & val1-H & val1-L & val2-H & val2-L \\
            NODE-REQ & npId-H & npId-L & item & arg1-H & arg1-L & arg2-H & arg2-L \\
            NODE-REP & npId-H & npId-L & item & arg1-H & arg1-L & arg2-H & arg2-L \\
            \bottomrule
        \end{tabular}
    \end{center}
\end{table}

Nodes do not react to attribute and user defined request messages when in operations mode. To configure a node, the node needs to be put into configuration mode. The \texttt{(OPS)} and \texttt{(CFG)} commands are used to put a node into configuration mode or operation mode. Not all messages are supported in operations mode and vice versa. For example, to set a new nodeId, the node first needs to be put in configuration mode. During configuration mode, no operational messages are processed.

\begin{table}[ht!]
    \begin{center}
        \caption{Node and Port Management}
        \begin{tabular}{|l|l|l|l|l|l|l|l|}
            \toprule
            \textbf{Opcode} & \textbf{Data1} & \textbf{Data2} & \textbf{Data3} & \textbf{Data4} & \textbf{Data5} & \textbf{Data6} & \textbf{Data7} \\
            \midrule
            OPS & npId-H & npId-L & & & & & \\
            CFG & npId-H & npId-L & & & & & \\
            \bottomrule
        \end{tabular}
    \end{center}
\end{table}


\section{Event Management}

The event management group contains the messages to configure the node event map and messages to broadcast an event and messages to read out event data. The (SET-NODE) with the item value to set and remove an event map entry from the event map is used to manage the event map. An inbound port can register for many events to listen to, and an outbound port will have exactly one event to broadcast. Ports and Events are numbered from 1 onward. When configuring, the portId \texttt{NIL} has a special meaning in that it refers to all portIds on the node.

\begin{table}[ht!]
    \begin{center}
        \caption{Event Management}
        \begin{tabular}{|l|l|l|l|l|l|l|l|}
            \toprule
            \textbf{Opcode} & \textbf{Data1} & \textbf{Data2} & \textbf{Data3} & \textbf{Data4} & \textbf{Data5} & \textbf{Data6} & \textbf{Data7} \\
            \midrule
            EVT-ON & npId-H & npId-L & evId-H & evId-L & & & \\
            EVT-OFF & npId-H & npId-L & evId-H & evId-L & & & \\
            EVT & npId-H & npId-L & evId-H & evId-L & arg-H & arg-L & \\
            \bottomrule
        \end{tabular}
    \end{center}
\end{table}

\section{DCC Track Management}

Model railroads run on tracks. Imagine that. While on a smaller layout, there is just the track, the track on a larger layout is typically divided into several sections, each controlled by a track node \textnormal{(centralized node or decentralized port)}. The system allows to report back the track sections status \textnormal{(in terms of occupied, free, and detecting the number of engines currently present)}. These messages allow the control of turnouts and monitoring of sections' status.

\begin{table}[ht!]
    \begin{center}
        \caption{DCC Track Management}
        \begin{tabular}{|l|l|l|l|l|l|l|l|}
            \toprule
            \textbf{Opcode} & \textbf{Data1} & \textbf{Data2} & \textbf{Data3} & \textbf{Data4} & \textbf{Data5} & \textbf{Data6} & \textbf{Data7} \\
            \midrule
            TON & npId-H & npId-L & & & & & \\
            TOF & npId-H & npId-L & & & & & \\
            \bottomrule
        \end{tabular}
    \end{center}
\end{table}

\section{DCC Locomotive Decoder Management}

Locomotive management comprises the set of messages that the base station uses to control the running equipment. To control a locomotive, a session needs to be established \texttt{(REQ-LOC)}. This command is typically sent by a cab handheld and handled by the base station. The base station allocates a session and replies with the \texttt{(REP-LOC)} message that contains the initial settings for the locomotive speed and direction. \text{(REL-LOC)} closes a previously allocated session. The base station answers with the \texttt{(REP-LOC)} message. The data for an existing DCC session can requested with the \texttt{(QRY-LOC)} command. Data about a locomotive in a consist is obtained with the \texttt{(QRY-LCON)} command. In both cases the base station answers with the \texttt{(REP-LOC)} message.

\begin{table}[ht!]
    \begin{center}
        \caption{DCC Locomotive Decoder Management}
        \begin{tabular}{|l|l|l|l|l|l|l|l|}
            \toprule
            \textbf{Opcode} & \textbf{Data1} & \textbf{Data2} & \textbf{Data3} & \textbf{Data4} & \textbf{Data5} & \textbf{Data6} & \textbf{Data7} \\
            \midrule
            REQ-LOC & adr-H & adr-L & flags & & & & \\
            REP-LOC & sId & adr-H & adr-L & spDir & fn1 & fn2 & fn3 \\
            REL-LOC & sId & & & & & & \\
            QRY-LOC & sId & & & & & & \\
            QRY-LCON & conId & index & & & & & \\
            \bottomrule
        \end{tabular}
    \end{center}
\end{table}

Once the locomotive session is established, the \texttt{(SET-LSPD)}, \texttt{(SET-LMOD)}, \texttt{(SET-LFON)}, \texttt{(SET-LOF)} and \texttt{(SET-FGRP)} are the commands sent by a cab handheld and executed by the base station to control the locomotive speed, direction and functions. \texttt{(SET-LCON)} deals with the locomotive consist management and \texttt{(KEEP)} is sent periodically to indicate that the session is still alive. The locomotive session management is explained in more detail in a later chapter when we talk about the base station.

\begin{table}[ht!]
    \begin{center}
        \caption{DCC Locomotive Decoder Management}
        \begin{tabular}{|l|l|l|l|l|l|l|l|}
            \toprule
            \textbf{Opcode} & \textbf{Data1} & \textbf{Data2} & \textbf{Data3} & \textbf{Data4} & \textbf{Data5} & \textbf{Data6} & \textbf{Data7} \\
            \midrule
            SET-LSPD & sId & spDir & & & & & \\
            SET-LMOD & sId & flags & & & & & \\
            SET-LFON & sId & fNum  & & & & & \\
            SET-LFOF & sId & fNum  & & & & & \\
            SET-FGRP & sId & fGrp  & data & & & & \\
            SET-LCON & sId & conId & flags & & & & \\
            KEEP & sId & & & & & & \\
            \bottomrule
        \end{tabular}
    \end{center}
\end{table}

Locomotive decoders contain configuration variables too. They are called CV variables. The base station node supports the decoder CV programming on a dedicated track with the \texttt{(REQ-CVS)}, \texttt{(REP-CVS)} and \texttt{(SET-CVS)} messages. The \texttt{(SET-CVM)} message supports setting a CV while the engine is on the main track. \texttt{(DCC-ERR)} is returned when an invalid operation is detected.

\begin{table}[ht!]
    \begin{center}
        \caption{DCC Locomotive Decoder Management}
        \begin{tabular}{|l|l|l|l|l|l|l|l|}
            \toprule
            \textbf{Opcode} & \textbf{Data1} & \textbf{Data2} & \textbf{Data3} & \textbf{Data4} & \textbf{Data5} & \textbf{Data6} & \textbf{Data7} \\
            \midrule
            SET-LSPD & sId & cv-H & cv-L & mode & val & & \\
            REQ-CVS  & cv-H & cv-L & mode & val & & & \\
            REP-CVS  & cv-H & cv-L & val  & & & & \\
            SET-CVS  & cv-H & cv-L & mode & val & & & \\
            \bottomrule
        \end{tabular}
    \end{center}
\end{table}

The SET-CVM command allows to write to a decoder CV while the decoder is on the main track. Without the RailCom channel, CVs can be set but there is not way to validate that the operation was successful.

\section{DCC Accessory Decoder Management}

Besides locomotives, the DCC standards defines stationary decoders, called accessories. An example is a decoder for setting a turnout or signal. There is a basic and an extended format. The \texttt{(SET-BACC)} and \texttt{(SET-EACC)} command will send the DCC packets for stationary decoders. Similar to the mobile decoders, there are POM / XPOM messages to access the stationary decoder via RailCom capabilities.

\begin{table}[ht!]
    \begin{center}
        \caption{DCC Accessory Decoder Management}
        \begin{tabular}{|l|l|l|l|l|l|l|l|}
            \toprule
            \textbf{Opcode} & \textbf{Data1} & \textbf{Data2} & \textbf{Data3} & \textbf{Data4} & \textbf{Data5} & \textbf{Data6} & \textbf{Data7} \\
            \midrule
            SET-BACC & adr-H & adr-L & flags & & & & \\
            SET-EACC & adr-H & adr-L & val & & & & \\
            \bottomrule
        \end{tabular}
    \end{center}
\end{table}

These commands are there for completeness of the DCC control interfaces. There could be devices that are connected via the DCC track that we need to support. However, in a layout control system the setting of turnouts, signals and other accessory devices are more likely handled via the layout control bus messages and not via DCC packets to the track. This way, there is more bandwidth for locomotive decoder DCC packets.

\section{RailCom DCC Packet management}

With the introduction of the RailCom communication channel, the decoder can also send data back to a base station. The DCC POM and XPOM packets can now not only write data but also read out decoder data via the RailCom back channel. The following messages allow to send the POM / XPOM DCC packets and get their RailCom based replies.

\begin{table}[ht!]
    \begin{center}
        \caption{RailCom DCC Packet management}
        \begin{tabular}{|l|l|l|l|l|l|l|l|}
            \toprule
            \textbf{Opcode}  & \textbf{Data1} & \textbf{Data2} & \textbf{Data3} & \textbf{Data4} & \textbf{Data5} & \textbf{Data6} & \textbf{Data7} \\
            \midrule
            SET-MPOM & sId & ctrl & arg1 & arg2 & arg3 & arg4 & \\
            REQ-MPOM & sId & ctrl & arg1 & arg2 & arg3 & arg4 & \\
            REP-MPOM & sId & ctrl & arg1 & arg2 & arg3 & arg4 & \\
            SET-APOM & adr-H & adr-L & ctrl & arg1 & arg2 & arg3 & arg4 \\
            REQ-APOM & adr-H & adr-L & ctrl & arg1 & arg2 & arg3 & arg4 \\
            REP-APOM & adr-H & adr-L & ctrl & arg1 & arg2 & arg3 & arg4 \\
            \bottomrule
        \end{tabular}
    \end{center}
\end{table}

The XPOM messages are DCC messages that are larger than what a CAN bus packet can hold. With the introduction of DCC-A such a packet can hold up to 15 bytes. The LCS messages therefore are sent in chunks with a frame sequence number and it is the responsibility of the receiving node to combine the chunks to the larger DCC packet.

\section{Raw DCC Packet Management}

The base station allows to send raw DCC packets to the track. The \texttt{(SEND-DCC3)}, \texttt{(SEND-DCC4)}, \texttt{(SEND-DCC5)} and \texttt{(SEND-DCC6)} are the messages to send these packets. Any node can broadcast such a message, the base station is the target for these messages and will just send them without further checking. So you better put the DCC standard document under your pillow.

\begin{table}[ht!]
    \begin{center}
        \caption{RRaw DCC Packet Management}
        \begin{tabular}{|l|l|l|l|l|l|l|l|}
            \toprule
            \textbf{Opcode}  & \textbf{Data1} & \textbf{Data2} & \textbf{Data3} & \textbf{Data4} & \textbf{Data5} & \textbf{Data6} & \textbf{Data7} \\
            \midrule
            SEND-DCC3 & arg1 & arg2 & arg3 & & & & \\
            SEND-DCC4 & arg1 & arg2 & arg3 & arg4 & & & \\
            SEND-DCC5 & arg1 & arg2 & arg3 & arg4 & arg5 & & \\
            SEND-DCC6 & arg1 & arg2 & arg3 & arg4 & arg5 & arg6 & \\
            \bottomrule
        \end{tabular}
    \end{center}
\end{table}

The above messages can send a packet with up to six bytes. With the evolving DCC standard, larger messages have been defined. The XPOM DCC messages are a good example. To send such a large DCC packet, it is decomposed into up to four LCS messages. The base station will assemble the DCC packet and then send it. 

\begin{table}[ht!]
    \begin{center}
        \caption{RRaw DCC Packet Management}
        \begin{tabular}{|l|l|l|l|l|l|l|l|}
            \toprule
            \textbf{Opcode}  & \textbf{Data1} & \textbf{Data2} & \textbf{Data3} & \textbf{Data4} & \textbf{Data5} & \textbf{Data6} & \textbf{Data7} \\
            \midrule
            SEND-DCCM & ctrl & arg1 & arg2 & arg3 & arg4 & & \\
            \bottomrule
        \end{tabular}
    \end{center}
\end{table}

\section{DCC errors and status}

Some DCC commands return an acknowledgment or an error for the outcome of a DCC subsystem request. The \texttt{(DCC-ACK)} and \texttt{(DCC-ERR)} messages are defined for this purpose.

\begin{table}[ht!]
    \begin{center}
        \caption{RRaw DCC Packet Management}
        \begin{tabular}{|l|l|l|l|l|l|l|l|}
            \toprule
            \textbf{Opcode}  & \textbf{Data1} & \textbf{Data2} & \textbf{Data3} & \textbf{Data4} & \textbf{Data5} & \textbf{Data6} & \textbf{Data7} \\
            \midrule
            DCC-ACK & & & & & & & \\
            DCC-ERR & code & arg1 & arg2 & & & & \\
            \bottomrule
        \end{tabular}
    \end{center}
\end{table}

\section{Analog Engines}

The messages defined for the DCC locomotive session management as outlined above are also used for the analog engines. An analog engine will just like its digital counterpart have an allocated locomotive session and the speed/dir command is supported. All other commands will of course not be applicable. The speed/dir command will be sent out on the bus and whoever is in control of the track section where the analog engine is supposed to be, will manage that locomotive. In the following chapters we will answer the question of how exactly multiple analog engines can run on a layout.

\section{Summary}

The layout system is a system of nodes that talk to each other. At the heart are consequently messages. The message format is built upon an 8-byte message format that is suitable for the industry standard CAN bus. Although there are many other standards and communication protocols, the CAN bus is a widely used bus. Since all data is encoded in the message, there is no reason to select another communication media. But right now, it is CAN.


