\chapter{The Base Station}

Take a deep breath. We are about to put together out first major LCS hardware module. The previous chapters introduced the message format and protocols and the core library for implementing the event system as well as the running equipment based on the DCC signal standards. Next, we took a closer look on the major hardware building blocks and power module designs. Just like the LCS core library allows to build a node specific firmware on top, the hardware building blocks are the foundation to build the required hardware modules. We also looked at how one would go after designing the node firmware in general. So here is the first and most important hardware module putting it all together. The base station. Every layout needs to have some kind of a base station that acts as the central place for layout control and signal generation.

Looking at the market, there are plenty of so called base stations. They typically offer support for several standards and communication protocols, such as DCC, mfx, LocoNet, a Can Bus, a S88 sensor bus, and so on. Most base station also have the power module directly integrated. They support the configuration of locomotive and stationary decoders. In short, a one stop all round solution. Their price range is around few hundred Euros. With the advent of Arduino, Raspberry and other controllers there are numerous do it yourself solutions. Just to name one, the \texttt{DCC++} Arduino base station with a motor shield as a power unit, gets you a base station for well under hundred Euros. The excellent work of the JMRI community to provide a \texttt{DCC++} interface for configuration software and other utilities to use this inexpensive base station hardware. The DCC-EX group extended and stabilized the original \texttt{DCC++} work for a wider range of controllers but also with new capabilities. There are many such great projects. The appendix provides some links and pointers to this work.

\section{Key Requirements}

Our base station needs to deliver the following capabilities. At first it needs to be able to assemble the DCC packets and generate the respective DCC hardware signals. This work is split into the base station producing the signal content and the power section driving the hardware. Furthermore, the base station needs to provide a way to manage several locomotive sessions. For each active session the current state of the locomotive is maintained and the DCC packets are produced. When a new locomotive session is established, a dictionary of locomotives could be consulted about the particular locomotive to get the initial function settings, and so on.

A base station implementing locomotive session and track management should also implement a serial command interface for managing a session or sending commands to a locomotive. Although not really necessary, it is very beneficial for testing and debugging. But also, programming DCC decoders will need some form of getting the configuration data to the decoder. There are great openSource tools out there which make use of an ASCII interface to send their commands. An example is DecoderPro from the JMRI teams. It features among other protocols the \texttt{DCC++} ASCII interface to send and receive commands. Our base station will therefore implement the relevant \texttt{DCC++} commands.

All configuration settings, such as the number of concurrent sessions or the current consumption limit for a track, should be available as attributes on the node itself and ports. A track should be represented by a port, so there will be a port for controlling the MAIN track output, and a port for controlling the PROG tack output.

Finally, the firmware for the base station could also host LCS management functions, such as a configuration database or display data about layout operations. This is not per se a function of the base station, but as each layout needs to have a base station and perhaps display high level status data, it is a convenient place to put central functionality there as well. this subject will be discussed in another chapter, this chapter will focus on the core base station features.

\section{Module hardware}

The base station is essentially based on a main controller board and a dual power module unit shown in previous chapters. We will use the PICO controller version and the dual H-Bridge building block. Add the RailCom detectors and stir the whole soup for a while. The extension connector of the base station would still export the I2C interface and the DCC signal produced by the base station. So, adding an I2C based extension board for display, switches, etc. is of course still possible. Here is the schematic for the base station. The individual building blocks should be familiar by now. The first page shows the main controller parts. Note that it needs fewer level shifters, as most of the signals are consumed internal to the board.

\begin{tikzpicture}[scale=0.9, transform shape]

    \draw[help lines, gray!50, dashed] (0,0) grid( 16,8);
    \node at (8,4) {picture};

\end{tikzpicture}

The Raspberry PI Pico has enough capacity to implement the CAN bus protocol directly using one of the cores. As a result, only the line driver is necessary to implement the CAN bus interface. There are the level shifters for the I2C bus and a few other external signals. The Power supply needs to have a method to detect that there is an USB cable connected to the PICO and ensure that there is no conflict between the power sources. Like any LCS node, the controller needs a non-volatile memory. The base station features two chip slots with each slot hosting an I2C type NVM with up to 128Kbytes. The key reason for such a high capacity is that a base station might store a lit of data for each engine that it can manage.

\begin{tikzpicture}[scale=0.9, transform shape]

    \draw[help lines, gray!50, dashed] (0,0) grid( 16,8);
    \node at (8,4) {picture};

\end{tikzpicture}

The next part shows the power module and RailCom detector unit. The power module exports the DCC signals via the external track power connectors and also as part of the LCS message bus connector connector. The track power extension connector is used by extension boards that directly use the H-Bridge output. A good example is an occupancy detector board which takes the DCC outputs and routes them to different sections, each equipped with a detectors for power consumption on the track. The power module unit features two identical channels. They are labelled "MAIN" and "PROG". Although the "PROG" channel would not need to deliver a high amperage, the dual H-Bridge is there anyway. And as said before, it would be nice to dynamically treat the PROG track as a type MAIN track too. Both channels therefore also feature a RailCom detector. Refer to the base station firmware chapter on what we actually do with a RailCom detector.

\begin{tikzpicture}[scale=0.9, transform shape]

    \draw[help lines, gray!50, dashed] (0,0) grid( 16,8);
    \node at (8,4) {picture};

\end{tikzpicture}

The base station will also offer an optional LocoNet interface. One day. LocoNet is very popular communication network for model railroads and there are a lot of devices such as Cab Handhelds that connect to the LocNet bus. Wouldn't it be nice to just connect these handhelds and alike via the LocoNet bus such that they can be used as well ? I guess it would. Right now, this is work in progress, but let's already reserve the the space on the board. Here is a first sketch of the interface.

\begin{tikzpicture}[scale=0.9, transform shape]

    \draw[help lines, gray!50, dashed] (0,0) grid( 16,8);
    \node at (8,4) {picture};

\end{tikzpicture}

Finally, there are the connectors and the power supplies. While the LCS node runs with 5V as before, the LocoNet interface will offer 12V on its bus to connected devices. This part is also optional if LocNot is not implemented. In addition to the basic connectors that we have as part of the LCS node design, there are also two power connectors for the two tracks.

\begin{tikzpicture}[scale=0.9, transform shape]

    \draw[help lines, gray!50, dashed] (0,0) grid( 16,8);
    \node at (8,4) {picture};

\end{tikzpicture}

Like the main controller, the monolithic base station board makes extensive use of SMD parts. While the previous boards have already been using passive SMD parts such as capacitors and resistors, this board also make use of SMD ICs. The exception is of course the Raspberry PI PICO board and the Dual H-Bridge IC L6205. The H-Bridge is a high power part and in case of a hardware problem it can easily be replaced as a DIP version.

\section{Base Station PCB}

The following picture shows the PCB for the monolithic base station. It is a 12cm by 10cm board, with the standard connectors in the usual place. As said, the LocoNet Interface is optional. Each LCS node needs at least one NVM chip. The second is optional too.

\begin{tikzpicture}[scale=0.9, transform shape]

    \draw[help lines, gray!50, dashed] (0,0) grid( 16,8);
    \node at (8,4) {picture};

\end{tikzpicture}

\section{Summary}

The base station, no matter how it is implemented, is a key component in any digital layout. It is primarily responsible for the locomotive session management and track signal generation. There could be designs where the base station is a device in a nice housing with a display, switches and so on. Other designs just put the LCS node somewhere on the layout. All these options will work just fine. And just like any other LCS node, the base station firmware offers through node and port attributes status and control functions.

The base station was also the first major LCS node with a considerable amount of hardware and software and thus went to several iterations and a considerable amount of learning curves. The very first version started with an Arduino Mega and Motor driver breakout board. The software was the original DCC++ software. The next versions actually were just a refactoring effort and work on the software. I learned a great deal on how all this works. The next base station version was a single board with all possible interfaces on an experimental PCB. It already featured CAN BUS, dual H-Bridges, and a controller, the Atmega 1284. This board served a long time for software development and further experiments. The next version of the base station was built with a main controller PCB described in the earlier chapters and a dual power unit, that also featured the RailCom detector. And again further software refactoring, refinement and learning. Finally, after a switch to the Raspberry Pi Pico controller, first on breadboard  then on PCBs, the base station shown in this chapter is the current version. The lessons learned proved to be very valuable for the overall concept development and hardware design. And in addition, there was a great deal of reading on chip specifications and, very importantly, the DCC standards. The appendix contains a list of material and web links. I highly recommend to read the electrical and DCC related standards published by the RailCommunity organization.

