\chapter{\textit{Layout System Configuration}}

// \textbf{note} under construction. A huge part in itself... what to do for a basic configuration ?

So far, the chapters concentrated on the LCS core library, extension firmware and the underlying hardware to implement the LCS nodes. From a firmware perspective the core element is the node with node attributes, ports and port attributes, an event system and a message interface for the odes to communicate to each other. Throughout the previous chapters configuration was essentially the setting of node, port and event data. This chapter will start to describe how that data is set, i.e. configured.

When it comes to layout system configuration there are two major philosophies. The first insist that a hardware module can be configured without the help of any external device such as a computer. These hardware modules typically offer some buttons and switches, some even a display, for entering the required configuration data through menus or sequences of how to push buttons during startup. This process is then called "teaching the node". Often it is recommended to configure before actually installed on the layout.

The other approach use a computer to configure. Especially larger layouts will have a graphical plan of the layout and offer numerous menus and options to configure the hardware modules even in the installed state. They also keep data bases of the layout configuration data for each node and all the running equipment.

\section{\textit{Requirements}}

\section{\textit{Layout Configuration Software}}

\section{\textit{Summary}}
