%-------------------------------------------------------------------------------------------------------
\chapter{RtLib Callbacks}

One key idea in LCS library message processing is the idea of a callback method to interact with the node firmware. The library inner loop function will continuously check for incoming messages, command line inputs and other periodic work to do. Most of this work is handled by the core library code itself transparently to the node firmware. For example, reading a port attribute from another node is done without any user written firmware interaction. There are other messages though that require the node firmware interaction. As an example, consider an incoming event. We check that there is port interested and if so, invoke a callback with the message and port information to handle the event. The same applies to the console command line handler and the generic loop callback. Since the library has complete control over the processing loop, the callbacks are essential to invoke other periodic work. Depending on the callback type, it is invoked before the action is taken or afterwards. For example, switching from configuration mode to operations mode, will first perform the switch and then invoke the bus management callback routine if there was one defined.

\section{General Callbacks}

The general callback routine invokes the registered handler with messages that concern the general working of the node. Those are for example \texttt{(RESET)}, \texttt{(BUS\_ON)}, \texttt{(BUS\_OFF)}, but also \texttt{(ACK)} and \texttt{(ERR)}.

\lstset{style=codesnippetstyle}
\begin{lstlisting}
// ... the busMgt msg handler routine
void busMgtMsgHandler( uint8_t *msgBuf ) {
	//... handle the cases of busMgt messages
}
...
// during module firmware initialization ...
lcsLib -> registerMsgHandler( busMgtMsgHandler )
\end{lstlisting}

\section{Node and Port Initialization Callback}

Once the library is initialized the various handlers can be registered and all other firmware specific initialization can be done. The last step is the call to the **run** method, which will never return. The very first thing the **run** method does after some internal setup is to invoke the node and port initialization callback if registered. The callbacks are also invoked whenever a node is restarted with the (RES-NODE) command or the (RESET) command for nodes and ports. The following code snippet shows how to register such a callback.

\lstset{style=codesnippetstyle}
\begin{lstlisting}
// ... the node init msg handler routine
void nodeInitHandler( uint16_t nodeId ) { ... }
...
// during module firmware initialization ...
lcsLib -> registerInitCallback( NIL_PORT_ID, nodeInitHandler )
\end{lstlisting}

Note that a portID or \texttt{NIL\_PORT\_ID} will refer to the node. Registering an initialization callback fro a port will just pass a non-nil portId instead. The port init callbacks are invoked in ascending portId order.

\section{Node and Port Request Reply Callback}

Node and port attributes can be queried from other nodes. The reply from sending a (QRY-NODE) command to the target node, the (REP-NODE) message, is passed back to the requesting firmware through the node request callback.

\lstset{style=codesnippetstyle}
\begin{lstlisting}
// ... the node query handler routine
void nodeReqHandler(    uint16_t nodeId, 
                        uint8_t portId, 
                        uint8_t item, 
                        uint16_t val1, 
                        uint16_t val2 ) { ... }
...
// during module firmware initialization ...
lcsLib -> registerReqRepCallback( nodeReqHandler );
\end{lstlisting}

The callback returns in addition to the arguments, the node and port ID of the replying node. Again, a portId of \texttt{NIL\_PORT\_ID} refers to a node item answer.

\section{Node and Port Control and Info Callback}

The nodeControl and nodeInfo routines offer callbacks for user defined items. There is a callback function for user defined control items and one for the info items.

\lstset{style=codesnippetstyle}
\begin{lstlisting}
uint8_t ( *infoHandler ) ( uint8_t portId, 
                            uint8_t item, 
                            uint16_t *arg1, 
                            uint16_t *arg2 ) { ... }

uint8_t ( *ctrlHandler ) ( uint8_t portId, 
                            uint8_t item, 
                            uint16_t arg1, 
                            uint16_t arg2 ) { ... }
...
// during module firmware initialization ...
lcsLib -> registerInfoCallback( portId, infoHandler );
lcsLib -> registerCtrlCallback( portId, ctrlHandler );
\end{lstlisting}

All the callback routines return a status code. When the item is not found or the arguments are not valid, the callback should return an error code. Any other status than \texttt{ALL\_OK} is passed back to the caller as the result of the nodeInfo or nodeControl method.

\section{Inbound Event Callback}

The event callback function is invoked when an event was received and the node has an inbound port that is interested in the event. The eventId / portId was previously configured in the event map. A port reaction to the incoming event can be configured to have a delay between the receipt of the event and the actual invocation of the port event callback routine. The callback function is passed the actual event information.

\lstset{style=codesnippetstyle}
\begin{lstlisting}
// ... the inbound event handler routine
void eventHandler ( uint16_t nodeId, 
                    uint8_t portId, 
                    uint8_t eAction,
                    uint16_t eId, 
                    uint16_t eData ) { ... }
...
// during module firmware initialization ...
lcsLib -> registerPortEventCallback( eventHandler )
\end{lstlisting}

If there is more than one port configured to react on the the incoming event, they are invoked in ascending order of portIds. The ***eAction*** parameter specifies whether the event is a simple ON/OFF event or a generic event with optional associated data. Note that only ports can react to events.

\section{Console Command Line Callback}

The LCS library implements a console command interface. Although not typically used during normal operations, it is very handy for tracking down firmware problems during development. Furthermore, troubleshooting in a layout is a good reason for having such an interface. As we will see in the hardware section, a simple serial data line or even an USB connector can be part of the module hardware. Simply connecting a computer to the node allows to query and control the node. Note, that this is also to some degree possible using the LCS bus messages.

In addition to the serial commands defined for the LCS  core library, the firmware programmer can implement an additional command interface. Any command not recognizes by the library is passed to the registered command line callback. The callback itself returns a status code about the successful command execution. Any status other than ALL-OK will result in an error message listed to the serial command device connected.

\lstset{style=codesnippetstyle}
\begin{lstlisting}
// ... the command line handler routine
uint8_t commandLineHandler( char *line ) { ... }
...
// during module firmware initialization ...
lcsLib -> registerCommandCallback( commandLineHandler )
\end{lstlisting}

Why implementing a serial command handler on top of the core library serial commands? The key reason is that a firmware programmer can add additional commands for firmware specific commands. Other than further debug and status commands, nodes such as the base station can implement an entire set of their own commands. A good example is our base station, which implements most of the \texttt{DCC++} serial command set. Configuring a DCC locomotive decoder can then be handled with decoder programming software such as the JMRI DecoderPro tool, which in turn issues \texttt{DCC++} commands as one option.

\section{DCC Message Callback}

The LCS Library defines a set of DCC related LCS messages to configure and operate the running equipment and track. These messages are typically used by cab handhelds and the base station, which is in charge to produce the DCC signals for the tracks. The DCC message callbacks are used to communicate these messages to the node firmware. The callback routines are all passed the message buffer. The following code snippet shows the declaration for a DCC type callback.

\lstset{style=codesnippetstyle}
\begin{lstlisting}
// ... the DCC message handler routine for DCC messages
void dccMsgHandler( uint8_t *msg ) { ... }
...
// during module firmware initialization ...
lcsLib -> registerDccMsgCallback( dccTrackMsgHandler )
\end{lstlisting}

\section{RailCom Message Callback}

Railcom is a concept for the DCC decoders to communicate back. DCC is inherently a broadcast protocol just like a radio station. There was no way to communicate back.  Railcom was design to allow for a decoder to send back data when the DCC channel is told to "pause". The chapter on the DCC subsystem will explain DCC and RailCom in greater detail. The Railcom Message callback is the function callback that will be invoked when a RailCom Messages is received.

\lstset{style=codesnippetstyle}
\begin{lstlisting}
// ... the Railcom message handler routine for DCC messages
void railComMsgHandler( uint8_t *msg ) { ... }
...
// during module firmware initialization ...
lcsLib -> registerRailComMsgCallback( dccTrackMsgHandler )
\end{lstlisting}

\section{LCS Periodic Task Callback}

The LCS core library attempts to handle as much as possible of message and event processing transparent to the user developed firmware. The core library ***run*** method, called last in the firmware setup sequence, will do the internal housekeeping and periodically scan for messages and serial commands. In addition, the run loop will also handle periodic activities outside the library. For example, a booster needs to periodically monitor the current consumption. The library therefore offers a callback registration function for periodic tasks. The example shown below registers a task to be executed every 1000 milliseconds.

\lstset{style=codesnippetstyle}
\begin{lstlisting}
// ... a periodic task to be registered
void aTask( ) { ... }
...
// during module firmware initialization ...
lcsLib -> registerPeriodicTask( aTask, 1000 );
\end{lstlisting}

The runtime library ***run*** routine never returns. All interaction between the library is done through previously registered callbacks and calls to the library from within those callbacks. It is also important to realize that a callback runs to completion. In other words, the library inner working is put on hold when executing a callback. For example, no further LCS messages are processed during callback execution. The same is true for the periodic tasks. It also means that one cannot rely on exact timing. Specifying for example a 1000 milliseconds time interval, could mean that the task is invoked later because of other tasks running for a longer period. A periodic task would however not run earlier than the specific interval. In summary, callback routines should therefore be short, quick and mist of all non-blocking.

Putting the library inner working on hold is however not true for functions that react on hardware interrupts. If there are interrupt routines for let's say a hardware timer, they will of course continue to take place. As we will see in the DCC track signal generation part of the base station, the interrupt driven signal generation is not impacted. Nevertheless, a firmware programmer needs to be aware that the order of callback invocation is fixed and that a callback runs to completion.

\section{Summary}

LCS callbacks are a fundamental concept in the core library. A firmware designer will write code that uses the core library functions to access the lower layers and callback functions that are invoked by the library to communicate back. Well, that is all there is a the core layer. Other than functions and callbacks, how can you access the library ? Wouldn't is be nice to have a simple interface to access the node data, set some options and simply test new hardware ? That is the subject of the next chapter.

%-------------------------------------------------------------------------------------------------------
\chapter{RtLib Command Interface}

???  explain the general concept ...

The primary communication method of the layout control system are LCS messages sent via the bus. In addition, each module that offers an USB connector or the serial I/O connector, implements also the serial command console interface. The interface is intended for testing and tracing purposes. LCS console commands are entered through the hardware module serial interface. 

Perhaps the most important command is the help command, which lists all available command and  their basic syntax.

\lstset{style=codesnippetstyle}
\begin{lstlisting}
<!?>
<#?>
\end{lstlisting}

Any command not recognized is passed to a command line handler....

<lcs-command-char [ arguments ] >


will be passed to the registered command call back function, if there is one registered. The following summary shows the available LCS serial commands. The appendix contains a detailed description of of the commands implemented by the LCS library.

\section{Configuration Mode Commands}

The configuration mode commands will place a node into either operations or configuration mode.

%|Command | Arguments | Operation |
%|:----|:-----|:-----|
%|!c | | enter node configuration mode |
%|!o | | enter node operations mode |

\section{Event Commands}

Event commands work with the event map. They add and remove an event, search the map for an event/port pair, or locally send an event to the node itself to test the event handling and so on.

%%|:----|:-----|:-----|
%|!a | eventId&nbsp; \[portId\] | add an event to the eventMap. If the portId is omitted, all eventMap %entries with a matching eventId will be removed.  |
%|!r | eventId&nbsp;\[portId\]| remove an event the eventMap. If the portId is omitted, all eventMap entries with a matching eventId will be removed. |
%|!f | eventId&nbsp; \[ portId \] | search an event in the eventMap. |
%|!e | mode&nbsp;nodeId&nbsp;eventId&nbsp;\[arg\] | simulate sending an event ( mode: 0 - ON, 1 - OFF, 2 - EVT ) |

\section{Node Map and Attributes Commands}

The node map and attribute map will examine and modify these maps.

%%|:----|:-----|:-----|
%|!n| item \[arg1\[arg2\]\]| lists a node or port map item.|
%|!N| item \[arg1\[arg2\]\]| sets a node or port map item.|
%|!v | attrId \[mode\] | list a global attribute. |
%|!V | attrId mode val | sets a a global attribute.|

\section{Send a raw Message}

For testing the message send mechanism, a command is available to send a raw data packet via the LCS bus.

%|Command | Arguments | Operation |
%|:----|:-----|:-----|
%|!B | byte1 [ byte2 .. byte8 ] | send a raw LCS message |

\section{List node status}

The "s" command will list a great detail on the node data. When debugging a node problem, this is perhaps the most useful command to see what is store locally.

%|Command | Arguments | Operation |
%|:----|:-----|:-----|
%|!s | \[level\] | list status at detail level, default is summary. ( 1 - ConfigDesc, 2 - NodeMap, 3 - %PortMap, 4 - EventMap, 6 - NVM Area, 7 - MEM Area ) |

\section{Driver commands}



What about the "xxx" commands? Well, they are used issue commands to the hardware drivers. We have not talked about them so far. This topic is presented when we know more about how the hardware is structured. Stay tuned.

\section{LCS message text format}

Just like the LCS core library accepts simple ASCII command strings, the LCS messages can also be transmitted as an ASCII text line. This is very useful for building communication gateways that transmit the message via another medium, such as an ethernet channel. There is a simple scheme for the ASCII representation of the message:

%```
%<$ data-byte-1 ... data-byte-n>
%

The message is enclosed in the "<" and ">" delimiters and the first character is the "xxx" sign. Up to 8 hexadecimal values written as "0xdd" follow, where "d" is a hexadecimal digit.

%%```cpp
%int lcsMsgToStr( char *msgStr, uint8_t *dataBuf, uint8_t dataBufLen );

%int strToLcsMsg( uint8_t *dataBuf, uint8_t dataBufLen, char *msgStr );
%```

Note: to be implemented. Perhaps to simple library routines to create an ASCII version of a LCS message and convert an ASCII string to an LCS message.

\section{Summary}

The command line interface provides a way to interact with a node at the command line level. This is very useful for initial testing new hardware and software debugging. All that is needed is a USB interface and a computer. As we will see in the main controller chapter, a USB or serial interface is also necessary for downloading new firmware to the boards. Besides that, this interface is normally not used during regular operations.
