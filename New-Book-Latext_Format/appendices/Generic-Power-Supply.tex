\chapterr{\textit{Appendix n - A generic power supply}}

Depending on the actual node hardware design, power is implemented in a variety of ways. A small handheld node would certainly draw its power from the LCS bus. A booster has a much higher power consumption requirement. The typical node needs in any case a 5V power, which can for example be drawn from a high power line on the layout. And as with every building block shown so far, there are many ways to Rome.

A power supply for a generic node could draw power from the LCD bus or from an external power line. Furthermore, some LCS nodes need a way to detect a power failure and perform any last second items before power is gone. The following schematic shows a power supply that allows for automatic switching between two inputs. It also features a power fail detection mechanism.


!\href{./Schematics/Schematic_LcsNodes-Building-Block-Generic-Power-Supply.png }{Schematic_LcsNodes-Building-Block-Generic-Power-Supply.png}

The right part of the schematic shows how the power input lines are switched depending what is connected. The voltage regulator itself is pretty much standard. Since the power line may have up to 24Volts, a switched regulator is a good choice. The right side features a power fail signal detection output and a capacitor to provide power for the last actions before power done. Naturally the timing depends on the actual power drawn by the board. When a power fail is detected, it is a good idea to immediately turn off power consuming devices and focus on the last items to do before power is gone. A good example is to save the last data items in the non-volatile storage.