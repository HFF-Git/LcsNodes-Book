\chapter*{The Base Station}
\addcontentsline{toc}{chapter}{The Base Station}

Take a deep breath. Over the next chapters we are about to put together our first major LCS hardware module. The previous chapters introduced the message format and protocols and the core library for implementing the event system as well as the running equipment based on the DCC signal standards. Next, we took a closer look on the major hardware building blocks and power module designs. Just like the LCS core library allows to build a node specific firmware on top, the hardware building blocks are the foundation to build the required hardware modules. We also looked at how one would go after designing the node firmware in general. So here is the first and most important hardware module putting it all together. The base station. Every layout needs to have some kind of a base station that acts as the central place for layout control and signal generation.

Looking at the market, there are plenty of so called base stations. They typically offer support for several standards and communication protocols, such as DCC, mfx, LocoNet, a Can Bus, a S88 sensor bus, and so on. Most base station also have the power module directly integrated. They support the configuration of locomotive and stationary decoders. In short, a one stop all round solution. Their price range is around few hundred Euros. With the advent of Arduino, Raspberry and other controllers there are numerous do it yourself solutions. Just to name one, the \texttt{DCC++} Arduino base station with a motor shield as a power unit, gets you a base station for well under hundred Euros. The excellent work of the JMRI community to provide a \texttt{DCC++} interface for configuration software and other utilities to use this inexpensive base station hardware. The DCC-EX group extended and stabilized the original \texttt{DCC++} work for a wider range of controllers but also with new capabilities. There are many such great projects. The appendix provides some links and pointers to this work.

Our base station needs to deliver the following capabilities. At first it needs to be able to assemble the DCC packets and generate the respective DCC hardware signals. This work is split into the base station producing the signal content and the power section driving the hardware. Furthermore, the base station needs to provide a way to manage several locomotive sessions. For each active session the current state of the locomotive is maintained and the DCC packets are produced. When a new locomotive session is established, a dictionary of locomotives could be consulted about the particular locomotive to get the initial function settings, and so on.

A base station implementing locomotive session and track management should also implement a serial command interface for managing a session or sending commands to a locomotive. Although not really necessary, it is very beneficial for testing and debugging. But also, programming DCC decoders will need some form of getting the configuration data to the decoder. There are great openSource tools out there which make use of an ASCII interface to send their commands. An example is DecoderPro from the JMRI teams. It features among other protocols the \texttt{DCC++} ASCII interface to send and receive commands. Our base station will therefore implement the relevant \texttt{DCC++} commands.

All configuration settings, such as the number of concurrent sessions or the current consumption limit for a track, should be available as attributes on the node itself and ports. A track should be represented by a port, so there will be a port for controlling the MAIN track output, and a port for controlling the PROG tack output.

Finally, the firmware for the base station could also host LCS management functions, such as a configuration database or display data about layout operations. This is not per se a function of the base station, but as each layout needs to have a base station and perhaps display high level status data, it is a convenient place to put central functionality there as well. This subject will be discussed in another chapter, this chapter will focus on the core base station features.

