\chapter{Cab Handheld Firmware}


This chapter will describe the firmware for our cab handhelds. 


 It directly connects via cable to the LCS bus and provides the generic elements to specify the locomotive to operate, set the speed and direction as well as the function keys. Implementing a base station and a handheld is all you would need to run an engine and finally see something for your hard work of building a layout system. The cab handheld described first is a board for developing the firmware. Nevertheless it can be used as a full functioning cab handheld. Later version will build upon the firmware but use a more handy form factor.

\section{Requirements}

A cab handheld needs to be able to control the loco. This implies that there is a local non-volatile memory that allows to remember locomotives once controlled. This way one can easily switch between a small set of locomotives and their characteristics. A display will show the actual state of cab handheld and allows together with the configuration buttons to configure the cab handheld. Looking at commercially available handhelds, they all seem to resemble TV controls. A numeric keyboard, some up and down buttons and the speed knob. ( No offense ). In all fairness, they are built to control not only the engines but also the rest of the layout.

But how about a cab handheld that features instead of all the functions to control an entire layout just the features to control an engine. Our cab handheld will have dedicated buttons and levers for let's say a horn or whistle, a bell, and so on. There are also configuration buttons, dedicated buttons and switches, and a very small set of buttons to map to loco specific functions. Furthermore, there is of course the rotary knob for setting the locomotive speed. The following figure shows a rough sketch of the cab handheld elements.

\begin{center}
    \begin{tikzpicture}[scale=0.9, transform shape]
        
     	\draw[help lines, gray!50, dashed] (0,0) grid(9,14);
     	
     	\node[ tsRoundedRectangle, 
                minimum width=9cm,
                minimum height=14cm,
                text width=3cm,
                text centered,
                fill=white!50] (display) at (4.5,7);

       	\node[ tsRoundedRectangle, 
                minimum width=3.5cm,
                minimum height=3.5cm,
                text width=3cm,
                text centered,
                fill=gray!30] (display) at (4.5,9.5) {Display};
        
        \node[ tsRoundedRectangle, 
                minimum width=1cm,
                minimum height=1cm,
                text width=1cm,
                text centered,
                fill=red!50] (horn) at (6.5,12.5) {Horn};
        
        \node[ tsRoundedRectangle, 
                minimum width=1cm,
                minimum height=1cm,
                text width=1cm,
                text centered,
                fill=blue!20] (menu) at (1.5,10.5) {Men};
                
      	\node[ tsRoundedRectangle, 
                minimum width=1cm,
                minimum height=1cm,
                text width=1cm,
                text centered,
                fill=blue!20] (up) at (7.5,10.5) {Up};

     	\node[ tsRoundedRectangle, 
                minimum width=1cm,
                minimum height=1cm,
                text width=1cm,
                text centered,
                fill=blue!20] (sel) at (1.5,8.5) {Sel};
                
    	\node[ tsRoundedRectangle, 
                minimum width=1cm,
                minimum height=1cm,
                text width=1cm,
                text centered,
                fill=blue!20] (down) at (7.5,8.5) {Dn};
                
     	\node[ tsRoundedRectangle, 
                minimum width=1cm,
                minimum height=1cm,
                text width=1cm,
                text centered,
                fill=red!50] (f1) at (1.5,6) {F1};
                
      	\node[ tsRoundedRectangle, 
                minimum width=1cm,
                minimum height=1cm,
                text width=1cm,
                text centered,
                fill=red!50] (f2) at (3.5,6) {F2};
                
      	\node[ tsRoundedRectangle, 
                minimum width=1cm,
                minimum height=1cm,
                text width=1cm,
                text centered,
                fill=red!50] (rev) at (1.5,4) {Rev};
                
      	\node[ tsRoundedRectangle, 
                minimum width=1cm,
                minimum height=1cm,
                text width=1cm,
                text centered,
                fill=red!50] (fwd) at (3.5,4) {Fwd};
                
      	\node[ tsRoundedRectangle, 
                minimum width=1cm,
                minimum height=1cm,
                text width=1cm,
                text centered,
                fill=red!50] (f3) at (5.5,6) {F3};
                
     	\node[ tsRoundedRectangle, 
                minimum width=1cm,
                minimum height=1cm,
                text width=1cm,
                text centered,
                fill=red!50] (f4) at (7.5,6) {F4};
                
       	\node[ tsRoundedRectangle, 
                minimum width=1cm,
                minimum height=1cm,
                text width=1cm,
                text centered,
                fill=red!50] (Bell) at (1.5,1) {Bell};
                
      	\node[ tsCircle,
         		minimum width=3cm,
                minimum height=3cm,
                text width=1cm,
                text centered,
                fill=red!50] (speed) at (6.5,2.5) {Speed};
         
    \end{tikzpicture}
\end{center}

Configuration and part of operation takes place with four buttons, which surround the screen display. The MENU button allows to toggle through the menus defined. To select a menu, the SELECT button is used. The menu toggle and select scheme can be nested. Within a menu screen, the MENU, UP and DOWN buttons are used screen specific and the SELECT button typically confirms the selected action. The direction buttons REV and FWD and the SPEED knob set the speed and direction of the locomotive or consist. F1 to F4 are four general buttons that can be mapped to special functions of the particular locomotive. The Horn and Bell button are rounding up the initial design.

The screen itself has also a common structure for all data displayed. 

\begin{center}
    \begin{tikzpicture}[scale=0.9, transform shape]
        
     	\draw[	help lines, gray!50, dashed] (0,0) grid(8,4);
     	
    	       	\node[	tsRectangle, 
     	 		minimum width=2cm,
                minimum height=1cm,
                text width=1cm,
                text centered,
                draw=gray,
                fill=none] (men) at (1,3.5) {Men};
                
      	\node[	tsRectangle, 
     	 		minimum width=2cm,
                minimum height=1cm,
                text width=1cm,
                text centered,
                draw=gray,
                fill=none] (up) at (7,3.5) {Up};
                
       	\node[	tsRectangle, 
     	 		minimum width=2cm,
                minimum height=1cm,
                text width=1cm,
                text centered,
                draw=gray,
                fill=none] (sel) at (1,0.5) {Sel};
                
       	\node[	tsRectangle, 
     	 		minimum width=2cm,
                minimum height=1cm,
                text width=1cm,
                text centered,
                draw=gray,
                fill=none] (down) at (7,0.5) {Dn};
                
        \node[	tsRectangle, 
     	 		minimum width=8cm,
                minimum height=4cm,
                text width=4cm,
                text centered,
                fill=none] (screen) at (4,2);
               
		\node at (4,3.5) {data field 1};
		\node at (4,0.5) {data field 2};
		\node at (4,2.5) {screen line 1};
		\node at (4,1.5) {screen line 2};
       
    \end{tikzpicture}
\end{center}

The screen display has several fields. The corner field match the buttons MENU, SEL, UP and DONW. The field width is four characters. The text shown is screen dependent. Typically the action of the four buttons is shown. Between the two corner fields on the top and on the button, there is a data field with up to eight characters. Finally, there are two screen lines in the center of the screen. 


\section{Cab Handheld Firmware}

Now that the development platform is in place, let's have a look at the firmware design. As you perhaps have guessed it, the hardware was already developed with a certain mode in mind. First of all, a cab handheld is nothing else than just another node on the LCS bus. The firmware sits on top of the LCS core library. In addition, there is another key library we have not talked about yet. A cab handheld and also any other device that allows for users to interact, needs software to work with buttons, encoders, displays and so on. This the tasks of the \textbf{UI Elements } library. We will look at this library in a later chapter in great detail.

// ??? note perhaps a picture ?

\begin{itemize}
\item SW architecture on top of the LCS library
\item UI is key to build a handheld
\item firmware to handle the buttons, switches, display, etc. Refer to UIElements.
\item issues LCS messages to the base station for speed, direction and functions
\item menu descriptions
\end{itemize}

\subsection{Concepts}

\begin{itemize}
\item a current cab and a stack of cabs to select from
\item base station has the ultimate data about a cab, loaded into the cab handheld
\item CabHandheld functions and DCC functions
\end{itemize}

\subsection{Screen Layout}

\begin{itemize}
\item display has 4 lines up to 16 characters. Two fonts
\item four navigation buttons, use top and bottom line, 8x8 font
\item two data lines between, 8x16 font.
\end{itemize}

\subsection{Screen Navigation}

\begin{itemize}
\item inherent in the UI Elements Screen Object design
\item MENU
\item SELECT
\item UP
\item DOWN
\end{itemize}

\subsection{Operate Screen}

\begin{itemize}
\item main screen, workhorse
\item speed, dir, functions
\end{itemize}

\subsection{Engine On/off Screen}

\begin{itemize}
\item for diesels only
\end{itemize}

\subsection{Engine Lights Screen}

\begin{itemize}
\item front and back lights...
\end{itemize}

\subsection{New Cab Screen}

There needs to be a way to set an engine cab number. The NEW CAB screen is used to enter a cab Id and engine type. We will display 4 digits and the engine type among we can toggle with the MENU button. The UP/DOWN buttons advance the current digit position. The encoder knob offers a fast way to scroll a digit. The high value digit allows to set an "S" instead of the number to indicate a short loco DCC address. The SELECT button completes the number entering and the current cab becomes this new cab. Note, that it would need to be explicitly saved.

\begin{itemize}
\item works on current cab setting
\end{itemize}

\subsection{Select Cab Screen}

A cab handheld maintains a stack of known cabs. That is cabs the handheld has used before and saved in the cab stack. This menu will toggle through them and select the new current cab. The UP/DOWN button is used to scroll around. In addition, the encoder knob allows to scroll a bit faster. The SELECT button will make the entry shown the current loco.

\begin{itemize}
\item SELECT scrolls through the cab stack and sets the cab selected as current cab.
\end{itemize}

\subsection{Save Cab Screen}

The current cab can be saved in the cab stack. This menu will toggle through them and select the cab slot for saving the current cab data. The UP/DOWN button is used to scroll around. In addition, the encoder knob allows to scroll a bit faster. The SELECT button will perform the action.

\begin{itemize}
\item SAVE scrolls through the cab stack and saves the current cab to this slot.
\item any previous entry used for the same cabId is cleared.
\end{itemize}

\subsection{Set DCC Function}

The DCC standard defines a list of 69 functions, F0 to F68.

\begin{itemize}
\item allows to set any DCC function ( F0 to F68 )
\item encoder knob for fast scrolling
\end{itemize}

\subsection{Config Cab Handheld Functions}

\begin{itemize}
\item connects a cab handheld function to a DCC function
\end{itemize}

\subsection{Options}

\begin{itemize}
\item all kinds of screen for configuration settings
\end{itemize}

\subsection{Diag}

\begin{itemize}
\item all kinds of screen for technical checks and tests
\end{itemize}

\subsection{Summary}

Phew. The cab handheld is another big step toward in operating a layout. After all, a layout control system without some form cab handhelds is not very useful. As said, there are many ways to build a cab. The design of UI elements and firmware was greatly influenced by a handheld called \textbf{\textit{Protothrottle}}. The concept of the four screen menu control buttons found its way into the UI Elements library. In addition to the general cab handheld, a cab handheld tailored toward a specific class of engines would be a great addition to operating that engine. The next section will present a diesel cab handheld that resembles a diesel cab stand from the 1950s.


\section{Summary}

\begin{itemize}
\item now we have a generic cab handheld and a diesel cab handheld.
\item one could come up with a steam or electric engine handheld. 
\item the firmware already goes a long way to quickly realize further cab handhelds.
\item would we ever build a handheld with other non-engine functionality... who knows... not right now.
\item would we build a version that can be integrated into a |\"Stellwerk\" table ? perhaps... just a main controller and an cab UI Elements extension
\end{itemize}

As always, there are many options to build a cab handheld. Although this version connects via cable to the LCS Bus, a wireless version is not hard to build. Having more buttons or fewer buttons, having a set of numeric keypad style input, are all quite valid options. It is a matter of what is preferred. Currently, the cab handheld will not offer any controls for accessories, such as turnouts. This is a subject better left to the layout control panels and controlling software. Our cab handheld favors the approach to model more of a locomotive control stand rather than a TV remote style handheld. Since this is a matter of taste and preference, go build your own.

There is certainly the option to connect commercially available handhelds. This would require to provide a gateway from let's say a LocoNet protocol based handheld to the LCS protocol. Refer to the base station part where is shows an optional LocoNet interface. Well, one day you should be able to connect such handhelds via the LocoNet bus. But right now the topic is on the backlog list.