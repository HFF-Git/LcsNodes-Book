\chapter{Inspiring work and links}

For me, work on this layout system started with getting my hands on an Arduino Uno and the proud setup of a blinking LED. Next, connecting a CAN bus shield allowed for two Arduino boards to talk to each other. I was hooked. There was the quick realization that this new world of "lego blocks" could be the basis of all kinds of projects, including projects for the model railroad. And sure enough, there are an overwhelming number of clubs and individuals working on the subject, willing to share their great expertise and work. 

This appendix just lists a few of the great web sites and projects that profoundly influenced the development of my layout control system. One I would like to point out is the IOTT YouTube series of Hans Tanner, which gives a wonderful introduction into electronics, concepts and applications for model railroading. In making all of my work public too, I hope to also give back to others that are underway in our great hobby.

\section{Standards}

The most relevant standard to read the DCC standard. There is the NMRA which owns it and their web site has all the relevant document. In Germany, the RailCommunity is also offering the DCC standard documents in close alignment with the NMRA.

\begin{table}[h!]
\centering
\begin{tabularx}{\textwidth}{lXl}
\toprule
\textbf{Organization} & \textbf{Link} & \textbf{Comment} \\
\midrule
NMRA & \url{http://www.nmra.org} & \\
\midrule
Rail Community & \url{http://www.RailCommunity.org} & \\
\bottomrule
\end{tabularx}
\caption{Relevant standards organizations}
\label{tab:standards}
\end{table}

\section{Projects}

There are many projects on the subject of layout control and DCC electronics. Below is a list of just a few of them. The list contains also the external libraries used in our runtime.

\begin{table}[h!]
\centering
\begin{tabularx}{\textwidth}{lXX}
\toprule
\textbf{Project} & \textbf{Link} & \textbf{Comment} \\
\midrule
DCC++ & \url{https://sites.google.com/site/dccppsite/} & DCC++ is a full-function open-source hardware and software system for the operation of DCC-equipped model railroads. \\
\midrule
DCC-EX & \url{https://dcc-ex.com} & DCC-EX is a team of dedicated enthusiasts producing open source DCC solutions for you to run your complete model railroad layout. \\
\midrule
MERG CBUS & \url{http://www.merg.org} & A system for comprehensive layout control based on a general purpose Layout Control Bus (LCB). \\
\midrule
can2040 & \url{https://github.com/KevinOConnor/can2040} & The CAN bus software implementation for the Raspberry Pi Pico. \\
\midrule
JMRI & \url{http://www.jmri.org} & The JMRI project is building tools for model railroad computer control. \\
\midrule
OpenDCC & \url{http://www.opendcc.de} & The OpenDCC web page contains a lots of useful information about model railroading and digital control. Definitively worth a visit. \\
\midrule
IOTT & \url{https://github.com/tanner87661} and also on YouTube & IOTT - the Internet of Toy Trains (Hans Tanner). \\
\midrule
Z21 & \url{https://pgahtow.de/w/Hauptseite} & Where I got the RailCom detector from. \\
\bottomrule
\end{tabularx}
\caption{Influential projects and resources}
\label{tab:projects}
\end{table}

\section{Tools}

\begin{table}[h!]
\centering
\begin{tabularx}{\textwidth}{lXl}
\toprule
\textbf{Tool} & \textbf{Link} & \textbf{Comment} \\
\midrule
Arduino & \url{http://www.arduino.cc} & - \\
\midrule
Raspberry PI Pico & \url{http://www.raspberrypi.org} & - \\
\midrule
EasyEDA & \url{http://easyeda.com/de} & Design tool for schematics and PCB layouts \\
\midrule
JLCPCB & - & Part of EasyEDA that manufactures PCB boards \\
\midrule
JMRI & \url{http://www.jmri.org} & - \\
\bottomrule
\end{tabularx}
\caption{Tools and platforms used}
\label{tab:tools}
\end{table}

\end{document}
