\chapter{Sensors and Actors}

Sensors and Actors are the eyes, ears and hands for any layout system. The requirements and options are numerous and the list of desired features needed is perhaps never complete. Sensors, the eyes and ears, are mainly the event producers. A block occupancy detection, a power overload detection, but also a push of a button on a layout control panel are good examples. The counterpart to sensors  are actors. Actors, the hands, are the family of LCS nodes and special hardware that control turnouts, signals and whatever else there is. This chapter will present the most common actors and sensor nodes found on a layout. Building upon the concept of main controller, base station, block controller and extensions board concepts, this part of the book will present how the pieces that make up a node can be put together from the basic building blocks.

With so many sensors and actors, a key requirement is to implement a concept where most of the functional parts can be used in different combinations. The chapter on hardware design already presented the main controller and extensions concept. A key reason for this concept were exactly the large variety of sensors and actors.

\begin{itemize}
\item main controller
\item block controller
\item base station
\end{itemize}

The main controller portion, common to all, will feature the processor, the message interface for the LCS bus, and the power supply for the board and the extensions. Some LCS nodes need a larger NVM storage. The main controller board will allow an optional installation of a NVM chip. Furthermore, the power supply, intended for generating the power for main controller board as well as the VCC power for the extensions, will also feature an optional installation of power failure detection. Before going into the detailed extension board designs, what boards would be need? Here is a brief look ahead of the boards to come.
\begin{itemize}
\item \textbf{Occupancy Detector}. The occupancy detector extension board is a companion to the block controller board. It will offer 4 channels, i.e. tracks that match the block controller output, with a set of detectors on each channel.
\item \textbf{Turnout}. The turnout extension board features a set of turnouts with frog polarization. Similar to the occupancy detector the boards is designed to go with the block controller board.
\item \textbf{Servo}. The servo extension board is a universal board, which can be connected to any controller board presented. It features a set of general servo outputs.
\item \textbf{GPIO}. The GPIO extension boards is a universal digital input / output board. It is typically used for input such as switches or push buttons a well as LEDs connected. Care has to be taken to match the the number of output consumers with the what the individual output pins can drive.
\item \textbf{Signal}. The signal extension bard is another general purpose board designed for driving the LED equipped signals. It is able to not only match the power requirements but also allow for dimming and / or blinking the individual lights.
\item \textbf{Relays}. The relay extension board is a general purpose board intended to drive a set of relays. The relays, which are not part of the board, can be driven with different voltage levels.
\item \textbf{Cab Control}. The cab handhelds as well as stationary cab control devices are also just an extension. There is no reason why this extension could for example not be connected to a base station and build a complete control station fort a smaller layout.
\item \textbf{Prototyping}. It is often useful to just do a quick sketch of a function desired for an extension board. So, how about an extension board with the extension address resolution logic and just room for building your own HW designs.
\end{itemize}

Extension boards implements the hardware for the particular sensor or actor type. All extension boards need to feature the connectors, so that more than one extension board can share the same controller board. However, only the first board will benefit from the availability of all extension bus signal lines. It is not a requirement that the second extension board will get the same signals. It is also not a requirement that an extension board does have the decoding logic described in the extension hardware design chapter. For example, the power module unit which together with main controller is a base station, does not have that logic. It expects to be connected directly to the main controller board. The power module board will however still route power, DCC and I2C signals. The I2C bus plays a key role in communicating with an I2C type extension board. Each I2C based extension board will have the extension board decoding logic, which is essential for proper addressing that board.