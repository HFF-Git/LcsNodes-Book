%--------------------------------------------------------------------------------
%
%
%
%--------------------------------------------------------------------------------

%--------------------------------------------------------------------------------
% Page layout settings
%
%--------------------------------------------------------------------------------
\geometry{a4paper, margin=1in, top=1in, bottom=1in, left=1in, right=1in, headheight=1in}
\addtolength{\parskip}{1.0ex}

\pagestyle{fancy}         
\fancyhf{}                     
\fancyhead[L]{\leftmark}       
\fancyfoot[LE,RO]{\thepage}         
\renewcommand{\headrulewidth}{0pt}
\renewcommand{\footrulewidth}{0pt}

%--------------------------------------------------------------------------------
% Force all "plain" pages (chapter starts, TOC, etc.) to use fancy layout
%
%--------------------------------------------------------------------------------
\makeatletter
\let\ps@plain\ps@fancy
\makeatother

%--------------------------------------------------------------------------------
% Formatting style for the "part" dividers.
%
%--------------------------------------------------------------------------------
\titleformat{\part}[display]
  {\normalfont\Huge\bfseries}
  {Part \thepart}
  {20pt}
  {\Huge}

%--------------------------------------------------------------------------------
% Chapter title settings 
%
%--------------------------------------------------------------------------------
\titleformat{\chapter}[hang]{\normalfont\huge\bfseries}{\thechapter}{2pc}{}
\titlespacing*{\chapter}{0pt}{0.1in}{0.2in}

%--------------------------------------------------------------------------------
% Table of Content settings 
%
%--------------------------------------------------------------------------------
\setlength{\cftsubsecindent}{1cm}
\setlength{\cftsubsubsecindent}{2cm}

%--------------------------------------------------------------------------------
% Table of contents options.
%
%--------------------------------------------------------------------------------
\setcounter{tocdepth}{1}
\setlength{\cftsecnumwidth}{3em}
\setlength{\cftsecindent}{0em} 

%--------------------------------------------------------------------------------
% Templates for Notes.
%
%--------------------------------------------------------------------------------
\newtcolorbox{fancyNote}{
  colback=yellow!10,
  colframe=orange!80!black,
  boxrule=0.8pt,
  arc=4pt,
  left=6pt,
  right=6pt,
  top=6pt,
  bottom=6pt,
  title=\textbf{Note}
}

\newcommand{\simpleNote}[1]{%
%  \noindent
  \textbf{Note:}\hspace{0.5em}%
  \hangindent=3em
  \hangafter=0
  #1\par
}

%--------------------------------------------------------------------------------
% Listing style for showing code snippets. This style is used for showing 
% short code sequences.
%
%--------------------------------------------------------------------------------
\lstdefinestyle{codesnippetstyle}{
    backgroundcolor=\color{gray!5}
    basicstyle={\ttfamily\small}, % Try \relsize{-2} if this is still too large
    keywordstyle=\color{magenta},
    identifierstyle={\ttfamily\color{gray}},
    numberstyle={\ttfamily\footnotesize\color{gray}},
    commentstyle={\ttfamily\color{teal}},
    showstringspaces=false,
    numbers=left,
    frame=single,
    texcl=false,
    alsoletter={\#},
    morekeywords={\#include, \#define, \#ifdef, \#endif, \#pragma},
    linewidth=0.9\textwidth,
    xleftmargin=0.1\textwidth,
    breaklines=true,
    breakatwhitespace=false,
    columns=fullflexible
}

%--------------------------------------------------------------------------------
% Common TIKZ styles for my pictures.
%
%--------------------------------------------------------------------------------
\tikzstyle{tsLargeBold} = [ 
    text=black, 
    font=\bfseries\large
]

\tikzstyle{tsWraptext} = [ 
    tsLargeBold, 
    text centered, 
    align=left
]

\tikzstyle{tsRectangle} = [ 
    rectangle,
    tsWraptext,               
    draw=black,
    line width=0.25mm]

\tikzstyle{tsRoundedRectangle} = [  
    tsRectangle,
    rounded corners
]

\tikzstyle{tsCircle} = [   
        circle,
        tsWraptext,
        draw=black,
        line width=1mm
    ]

\tikzstyle{tsEllipse} = [   
    ellipse,
    tsWraptext,
    draw=black,
    line width=1mm
]
