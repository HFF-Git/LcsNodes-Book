%--------------------------------------------------------------------------------
% To make our life easier, here are styles defined for the book.
%
%--------------------------------------------------------------------------------

%--------------------------------------------------------------------------------
% General Page layout settings.
%
%--------------------------------------------------------------------------------
\geometry{	a4paper, 
			margin=1in, 
			top=1in, 
			bottom=1in, 
			left=1in, 
			right=1in, 
			headheight=1in }
			
\addtolength{\parskip}{1.0ex}

\pagestyle{fancy}         
\fancyhf{}                     
\fancyhead[L]{\leftmark}       
\fancyfoot[LE,RO]{\thepage}         
\renewcommand{\headrulewidth}{0pt}
\renewcommand{\footrulewidth}{0pt}

%--------------------------------------------------------------------------------
% Force all "plain" pages (chapter starts, TOC, etc.) to use fancy layout
%
%--------------------------------------------------------------------------------
\makeatletter
\let\ps@plain\ps@fancy
\makeatother

%--------------------------------------------------------------------------------
% Formatting style for the "part" dividers.
%
%--------------------------------------------------------------------------------
\titleformat{\part}[display]
  {\normalfont\Huge\bfseries}
  {Part \thepart}
  {20pt}
  {\Huge}

%--------------------------------------------------------------------------------
% Chapter title settings 
%
%--------------------------------------------------------------------------------
\titleformat{\chapter}[hang]{\normalfont\huge\bfseries}{\thechapter}{2pc}{}
\titlespacing*{\chapter}{0pt}{0.1in}{0.2in}

%--------------------------------------------------------------------------------
% Table of Content settings 
%
%--------------------------------------------------------------------------------
\setlength{\cftsubsecindent}{1cm}
\setlength{\cftsubsubsecindent}{2cm}
\setcounter{tocdepth}{1}
\setlength{\cftsecnumwidth}{3em}
\setlength{\cftsecindent}{0em} 

%--------------------------------------------------------------------------------
% Tables format.
%
%--------------------------------------------------------------------------------
\keepXColumns
\newcolumntype{Y}{>{\raggedright\arraybackslash}X}

%--------------------------------------------------------------------------------
% Notes format.
%
%--------------------------------------------------------------------------------
\newtcolorbox{fancyNote}{
  colback=yellow!10,
  colframe=orange!80!black,
  boxrule=0.8pt,
  arc=4pt,
  left=6pt,
  right=6pt,
  top=6pt,
  bottom=6pt,
  title=\textbf{Note}
}

\newcommand{\simpleNote}[1]{
%  \noindent
  \textbf{Note:}\hspace{0.5em}
  \hangindent=3em
  \hangafter=0
  #1\par
}

%--------------------------------------------------------------------------------
% Smaller item lists.
%
%--------------------------------------------------------------------------------
\newenvironment{smallGapItemize}{
  \begin{itemize}[itemsep=4pt, parsep=0pt, topsep=2pt, partopsep=0pt]
}{\end{itemize}}

%--------------------------------------------------------------------------------
% Listing style for showing code snippets. This style is used for showing 
% short code sequences.
%
%--------------------------------------------------------------------------------
\lstdefinestyle{codesnippetstyle}{
    backgroundcolor=\color{gray!5},
    basicstyle={\ttfamily\small}, % Try \relsize{-2} if this is still too large
    keywordstyle=\color{magenta},
    identifierstyle={\ttfamily\color{gray}},
    numberstyle={\ttfamily\footnotesize\color{gray}},
    commentstyle={\ttfamily\color{teal}},
    showstringspaces=false,
    numbers=left,
    frame=single,
    texcl=false,
    alsoletter={\#},
    morekeywords={\#include, \#define, \#ifdef, \#endif, \#pragma},
    linewidth=0.9\textwidth,
    xleftmargin=0.1\textwidth,
    breaklines=true,
    breakatwhitespace=false,
    columns=fullflexible,
    keepspaces=true,
    tabsize=4
}

\lstdefinestyle{codesnippet-nobox}{
    style=codesnippetstyle,   % inherit all other settings
    frame=none,               % remove the frame
    backgroundcolor=,         % remove background
    xleftmargin=0pt,          % optional: remove left margin offset
    linewidth=\textwidth      % optional: use full width
}

%--------------------------------------------------------------------------------
% Common TIKZ styles for my pictures.
%
%--------------------------------------------------------------------------------
\tikzstyle{tsLargeBold} = [ 
    text=black, 
    font=\bfseries\large
]

\tikzstyle{tsWraptext} = [ 
    tsLargeBold, 
    text centered, 
    align=left
]

\tikzstyle{tsRectangle} = [ 
    rectangle,
    tsWraptext,               
    draw=black,
    line width=0.25mm]

\tikzstyle{tsRoundedRectangle} = [  
    tsRectangle,
    rounded corners
]

\tikzstyle{tsCircle} = [   
        circle,
        tsWraptext,
        draw=black,
        line width=1mm
    ]

\tikzstyle{tsEllipse} = [   
    ellipse,
    tsWraptext,
    draw=black,
    line width=1mm
]

%--------------------------------------------------------------------------------
% Instruction format.
%
% We have a rectangle field of 16cm width. Positions start left with 0 and end
% at 16 on the right. The first parameter to the macro is the position of the 
% separators. The second parameter is a list of labels to put into the fields 
% created by the separators. A bit position is 0.5 units.
%
%--------------------------------------------------------------------------------
\newcommand{\drawInstrFormat}[2]{%
  	
  	\begin{tikzpicture}[scale=0.7, transform shape]
    	% Outer box
    	\node[	tsRectangle, 
    			minimum width=16cm, 
    			minimum height=1cm, 
    			fill=white!50] at (8,1) {};

    	% Initialize
    	\pgfmathsetmacro{\last}{0}

   	 	% First split labels into individual macros
    	\foreach [count=\i] \l in {#2} {%
      		\expandafter\xdef\csname label@\i\endcsname{\l}%
    	}

    	% Iterate over positions
    	\foreach \p [count=\i] in {#1} {%
      		
      		\ifnum\i>1
        
        		% Width of field
        		\pgfmathsetmacro{\width}{\p-\last}
        
        		% Center of field
        		\pgfmathsetmacro{\center}{(\p+\last)/2}
        
        		% Convert width to integer bit size (0.5 cm = 1 bit)
        		\pgfmathtruncatemacro{\bits}{round(\width*2)}

        		% Vertical separator line
        		\draw[-] (\p,0.5) -- (\p,1.5);

        		% Place width number
        		\node at (\center,0.25) {\bits};

        		% Place corresponding label
        		\node at (\center,1) {\csname label@\the\numexpr\i-1\relax\endcsname};
      
      		\fi
      	\xdef\last{\p}
    }%
    
  	\end{tikzpicture}%
}

%--------------------------------------------------------------------------------
% Instruction format for a 64-bit word. It still has 16 unit width and the same 
% display logic as the 32-bit counterpart. A bit position is 0.25 units.
%
%--------------------------------------------------------------------------------
\newcommand{\drawInstrFormatSixtyFour}[2]{%
  	\begin{tikzpicture}[scale=0.7, transform shape]
    	% Outer box
    	\node[tsRectangle, 
    		minimum width=16cm, 
    		minimum height=1cm, 
    		fill=white!50] at (8,1) {};

    	% Initialize
    	\pgfmathsetmacro{\last}{0}

    	% Fully expand label list before the foreach
		\edef\expandedLabels{#2}%
		\foreach [count=\i] \l in \expandedLabels {%
  			\expandafter\xdef\csname label@\i\endcsname{\l}%
		}

    	% Iterate over positions
    	\foreach \p [count=\i] in {#1} {%
      		\ifnum\i>1
        		\pgfmathsetmacro{\width}{\p-\last}
        		\pgfmathsetmacro{\center}{(\p+\last)/2}

        		% Adjust bit calculation for 64-bit word (1 unit = 4 bits)
        		\pgfmathtruncatemacro{\bits}{round(\width*4)}

        		% Draw vertical separator
        		\draw[-] (\p,0.5) -- (\p,1.5);

        		% Place width number and label
        		\node at (\center,0.25) {\bits};
        		\node at (\center,1) {\csname label@\the\numexpr\i-1\relax\endcsname};
      		\fi
      		\xdef\last{\p}
    	}%
  	\end{tikzpicture}%
}

%--------------------------------------------------------------------------------
% Page description for instructions and command layout style.
%
%--------------------------------------------------------------------------------
\newenvironment{iPageDescEntry}[1]{%
  \par\vspace{0pt}% <--- forces vertical mode
  \noindent
  \begin{minipage}[t]{0.22\textwidth}
    \textbf{\large\textcolor{gray}{#1}}%
  \end{minipage}%
  \hfill
  \begin{minipage}[t]{0.74\textwidth}
    \setlength{\parskip}{1em}  
    \setlength{\parindent}{0pt} 
}{\end{minipage}\par\vspace{0.75em}}


%--------------------------------------------------------------------------------
% Page description layout style for instruction operation description.
%
%--------------------------------------------------------------------------------
\newenvironment{iPageDescOpEntry}[1]{%
 \par\vspace{0pt}% <--- forces vertical mode
	\noindent
  	\begin{minipage}[t]{0.22\textwidth}
    	\textbf{\large\textcolor{gray}{#1}}%
  	\end{minipage}%
  	\hfill
  	\begin{minipage}[t]{0.74\textwidth}
    	\setlength{\parskip}{1em}  
    	\setlength{\parindent}{3pt} 
    	\footnotesize
    	\begin{minipage}[t]{0.97\textwidth}%
	}{%
	
	\end{minipage}%
 	\end{minipage}\par\vspace{0.75em}%
}

%--------------------------------------------------------------------------------
% Instruction description pages use these layout definitions.
%
%--------------------------------------------------------------------------------
\lstdefinestyle{iPageOpStyle}{
    basicstyle=\ttfamily\footnotesize,
    showstringspaces=false,
    keepspaces=true,
    breaklines=true,
    columns=fullflexible,
    frame=none,
    aboveskip=0.6\baselineskip, 
    belowskip=0.6\baselineskip,
    xleftmargin=0.025pt,
    xrightmargin=0pt,
    linewidth=\dimexpr0.95\textwidth\relax,
    floatplacement=H,                      
}
