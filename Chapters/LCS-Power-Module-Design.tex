\chapter{Power Module Design}

LCS hardware modules generally consist of two parts. The first is the main controller portion where the processing and message interfacing is done. The second part typically implements special functions to complete the LCS node. One such function is the power module for the DCC subsystem. This chapter will present the basic components necessary for building the power generation part for a base station or booster and block controller module.

\section{DCC Track Power Modules}

A layout control system needs a form of DCC signal generator. They can be directly integrated with the base station to form one hardware module or be at the heart of a separated booster module. Depending on the model railroad scale, there are different voltages and maximum current considerations. A Gauge N layout has a different power requirement than a Gauge G layout. The NMRA standard defines the voltage ranges for the different gauges. Our assumption is that the layout system has a form of power supply that delivers the voltage and current requirements to the power module. For convenience, all hardware module node local power supplies for the controller and other chips should be able to handle a voltage range between 7V and 24V when drawing power form this power supply. Switching supplies are quite interesting, they do not produce that much heat compared to their analog counterparts. The main task of the DCC track power module is to provide the DCC signal to the track. DCC signals are square wave signals with a defined duty cycle period. A duty cycle of 58 microseconds represents a "DCC one", a duty cycle of 116 microseconds a "DCC zero" bit. A typical solution to this task is a H-Bridge design. The following figure shows the H-Bridge signal convention for LCS power modules.

\begin{tikzpicture}[scale=0.9, transform shape]

    \draw[help lines, gray!50, dashed] (0,0) grid( 16,8);
    \node at (8,4) {picture};

\end{tikzpicture}

While a power module can vary in capacity and technology components, our logical design expect to send the same digital signals to a power module, no matter what it's design. It is a key requirements to derive from the DCC signal on the bus connector the state "+", "-" and "short circuit". The latter is needed for the cutout option. There is also the need to detect a logic "one" on both signal inputs and interpret this as setting the bridge to a high impedance state. This requirement comes primarily from the capability to also issue PWM signals, which will depending in the direction alternate between a positive or negative voltage and the high impedance state. There is a whole family of H-Bridge ICs and breakout boards. If a particular H-Bridge IC does not map the inputs to the table below, some gate logic needs to be added. The following table depicts the power module digital input management signals. The table below fits the L6205 H-Bridge nicely.

\begin{longtable}{@{}|l|l|l|p{0.4\linewidth}|@{}}
    %\caption{Bus Connector Pins}
    \toprule
    \textbf{Enable} & \textbf{DCC-Sig-1} & \textbf{DCC-Sig-2} & \textbf{Meaning} \\
    \midrule
    \endfirsthead
    \toprule
    \textbf{Enable} & \textbf{DCC-Sig-1} & \textbf{DCC-Sig-2} & \textbf{Meaning} \\
    \midrule
    \endhead
    \midrule
    \multicolumn{2}{r}{\textit{Continued on next page}} \\
    \midrule
    \endfoot
    \bottomrule
    \endlastfoot
    1 & 0 & 0 & Cutout period. The track will be short circuited. Out1 and Ou2 are connected. \\
    \midrule
    1 & 1 & 1 & The track is polarized "+". Out1 is connected to VS, Out2 is is connected to GND. \\
    \midrule
    1 & 0 & 1 & The track is polarized "-". Out1 is connected to GND, Out2 is is connected to VS. \\
    \midrule
    X & 1 & 1 & The track is put into high impedance, i.e. not connected. \\
    \midrule
    0 & X & X & The track is put into high impedance, i.e. not connected. \\
\end{longtable}

The standards also specify which side of the track is the positive side, i.e. OUT1 in the table above. The right hand track side, usually connected with a red cable is the positive side, the left track side, connected with a black cabe, is the negative side. If the H-Bridge is enabled, sending a "DCC One" will mean to raise the digital output of the controller port DCC-SIG-1 to HIGH and controller port DCC-SIG-2 to LOW for 58 microseconds, followed by the reverse bit setting for another 58 microseconds. The DCC packet is broken down, bit by bit and the digital signal is produced. The H-Bridge hardware then takes the zero and ones and essentially reverses the track polarity accordingly to digital zeroes and ones.

If the power bridge is used for the PWM mode, a "FWD" means to raise  the DCC-SIG-1 to HIGH and the controller port DCC-SIG-2 to LOW for the active part of the duty cycle length. The remainder of the duty cycle, the port signals will be both set to a HIGH, which puts the bridge into the high impedance state. Not all H-Bridge ICs follow the same control signal level standard. Any difference in H-Bridge control signals inout logic need to be masked accordingly. The following figure shows the H-Bridge signal standard. All power modules and connectors to the track follow this convention.

All power modules are expected to deliver a voltage proportional to the power consumption of the H-Bridge. This is typically done by putting a shunt resistor between ground and the lower part of the H-bridge. For a better accuracy, the rather low voltage signal is amplified before delivered to the analog input of a controller. Note that most H-Bridge ICs offer short circuit and over temperature protection already as a built-in feature. Based on the analog voltage signal analysis, current protection features such as sending a power overload event can be implemented.

Power consumption measurement is necessary for another important DCC requirement. A DCC decoder in programming mode will raise its power consumption to acknowledge an operation with a raise by about 60mA. This short rise needs to be detected as well. It requires to calibrate the actual power consumption of the decoder, build a base of typical consumption and then detect the temporary raise. With the basics in place now, the next section will show different designs for a power module using the L6205 chip for the implementation. As always, there are other H-Bridges and also breakout boards that could also be used as the heart of a power module building block.

\section{Dual Power Module - L6205}

The representative schematic for a power unit uses the L6205 dual H-Bride IC for a dual power module unit. There is a serial resistor on bridge ground side to measure the current consumption. The voltage drop over this resistor is amplified and will be passed to an analog controller pin. Both bridges deliver up to 2.8A, which is sufficient for scales up to HO Scale.

\begin{tikzpicture}[scale=0.9, transform shape]

    \draw[help lines, gray!50, dashed] (0,0) grid( 16,8);
    \node at (8,4) {picture};

\end{tikzpicture}

When building a base station, one h-bridge is used or the main track and the other for the programming track. The programming track that actually would need a much lower current, why use a design with two equally powered H-bridges? One answer is that you can get such an ICs with two H-Bridges inside. The other answer is that a design with two equal H-Bridges would allow for an interesting feature. Imagine you could feed the main track signal to both the MAIN and the PROG track. A locomotive could drive under its own power onto the PROG track, which is acting as a MAIN track section. Then the DCC signal is switched back from MAIN to PROG and the locomotive configuration can begin. Now, the same could also be accomplished with some relay based logic, but wouldn't this be an elegant approach? More on this idea in the base station chapter.

\section{Mono Power Module - L6205}

The L6205 dual bridge can also be used for a DCC booster hardware module with a higher amperage output. By now, the basic parts of the schematic shown below should be familiar at a high level. The L6205 chip allows to combine the two H-Bridges to deliver up to 5.6Amps. All else is fairly identical to the dual design discussed before.

\begin{tikzpicture}[scale=0.9, transform shape]

    \draw[help lines, gray!50, dashed] (0,0) grid( 16,8);
    \node at (8,4) {picture};

\end{tikzpicture}

There is a smaller cousin, the L6225, which delivers two times 1.4 Amps or 2.8 if the two bridges are combined. The electrical control signals and the Pin layout are identical, so it is a good candidate for a smaller mono or dual power module.

\section{Power Module - Breakout Boards}

There are a lot of power module boards readily available. There is a popular Arduino shield version built around the L298 chip that can directly put on top of an Arduino UNO or MEGA. Just to name one. There are also breakout boards that deliver really high power levels, easily up to 10 or more Amps, at a very low cost. There is one very popular bridge out there built upon the BTS7960 half bridge, which is rated up to 30A. There is even a Arduino UNO shield available using the BTS7960 chips. Well, 30A is perhaps a bit too much for a model railroad unless you want to weld engines to the tracks in case of a short circuit. Depending on the breakout board used, some "glue" logic to match our DCC signal standard and a current consumption measurement logic needs to be added. The appendix has a section that describes the PCB layout for an empty board with just the connectors. This board can be used to piggyback a power module breakout board on top.

\section{Summary}

This chapter presented a basic track power module designs. It contains a H-Bridge and a means to return a voltage proportional the current consumption. Depending on the model scale and the layout size, power modules are found in base stations, boosters and block controllers. For interoperability, a power module building block is expected to accept the control signals commands via the digital control lines defined. Any new H-Bridge design needs to make sure that it supports the defined control signals.
