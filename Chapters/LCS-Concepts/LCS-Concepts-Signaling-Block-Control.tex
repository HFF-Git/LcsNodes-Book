\chapter{Signaling Block Control}

The previous chapters got us to understand the LCS concepts, to build LCS nodes and power units on different hardware platforms. It all resulted in building an essential piece necessary for any layout, the LCS base station. So far so good. The next big part to address is to build hardware and software to control devices such as signal, turnouts, and so on. Also, we would need to get feedback from the layout where trains exactly and what the currently do. After all, a trains needs to safely get from A to B. Before diving into block controllers, sensors and actors, here is a brief chapter on where the journey will go.

Safely getting a train from point A to B is the basic unit of work for a railway. In contrast to let's say a car, trains run on orders and there is a planned schedule. It is a planned movement. A train is also different to a car in that it cannot react rather quick to events such as a blocking signal. Try stopping a 15000 ton freight train half mile long. As a consequence, there is a concept of train authorization and train movement. The authorization represents the train order and the train movement gets the train safely from A to B obeying the rules of the signaling system. In the early days, there was one train moving from A to B and the authorization instance just authorized the train movement from A to B. A follow-on train moving from station A to station B had to wait for the complete route cleared over many miles as for a train moving in the opposite direction on a single line. With rising traffic the throughput was limited and new concepts were needed. One obvious solution was higher train speed and using two lines.

To increase track utilization, the real railways developed the concept of \textbf{signaling block control}. A route is divided into several blocks. A block has signals that indicate the state of the block ahead, if the block is occupied, the signal tells the following train to stop until the train ahead cleared the block. This way trains are protected from running into each other. In the early days the block signals where controlled manually by man living with their families in small houses built along the line, communicating the status of their block to their neighbor blocks by Telegraph. Trains had to carry red flags or lights on the last car so that man watching the passing train was sure that no car has been left in his block due to a broken coupler. Today automated block occupancy detectors are used.

With increased speed however, the status of the block signal has to be communicated to the engineer ahead, so he can  securely on the block signal. In Europe this is done by a breaking signal usually one kilometer ahead from the main signal. In the US the status of the block following the actual block is indicated by a multi aspect signal so that the engineer is alerted to expect a red signal after the green one right ahead. With modern state of the art systems the physical signals along the route have been replaced of the status of the line ahead directly to the engineers cab by radio transmission.

So in general, a route from A to B is a series of blocks each governed by a signal, which shows the state of the block ahead. A train will not enter a block that is occupied by another train. Since a train movement is a planned movement, there needs to be an authorization system that locks the rote and sends the train. Sounds straightforward for a single line between A and B. But consider a block with a turnout. In that case there are two follow on blocks, depending on the setting of the turnout. Inside a station or freight yard, this can get even more complex. On the other hand, a lot of safety is introduced into the operations of a layout. Typically, the track lines between stations are under a block control while a station or freight yard form one block under manual control. On a model railroad, such a block system does not only prevent trains from crushing into each other on track segments which cannot be seen by the operator in tunnels or "shadow yards“, but also allows a basic "automation mode" for both DCC ad DC where you just can watch one train following the other, especially for European stile main lines with a dedicated track for each direction.

This chapter will present the signaling block concepts and gives an overview how they are implemented for the layout control system at a high level. The appendix contains further references to detailed literature about the subject. A later chapter will discuss the block controller hardware and firmware.

\section{Requirements}

As discussed already, a layout is divided into blocks, and a block further divided into subsections. Once the layout is built and configured the relation between the blocks, i.e. which blocks to enter from and to leave top are fixed. There is a static relation and a dynamic one, called routes. Consider a block with an entry and a turnout at the other end. The entry block is fixed and the two exit block too. The actual turnout setting determines which is the follow-on block. Since a block can possibly be entered from both directions, the fundamental block is a piece of track with an optional turnout at each end.

Blocks needs to be uniquely identifiable across the layout, so the configuration tool can describe the overall layout in terms of blocks as the unit of control. The block has a state. At a high level, there are free, locked and occupied blocks. Consider a train in Block A on route from block A to B to C. Block A is marked "occupied". The follow on block B will be marked "locked" and block C is "free". Once the train has left block A into block B, block A will be marked free after a short delay, block B is "occupied" and block C is "locked". Trains always look ahead two blocks, if not more.

Each block has attributes for current speed and direction of a train occupying the block. Each block further needs an attribute that specifies the entry speed when moving into this block. A speed of zero means that this block is occupied and a follow-on train cannot enter the block. If for example block C in our example is occupied, a train being in block A needs to enter block B at reduced speed. The algorithms will be described in a further section of this chapter.

A block is divided into one more subsections. While they all share the same power module, power consumption can be measured to determine in which subsection the current train is. A block should have at least two subsection if unidirectional, and three, better four, when trains can go in both directions.

A block is guarded by signals. They indicate information to the locomotive engineer about the state of the blocks ahead. The setting of the signals is a consequence of the block events broadcasted among the blocks about trains, speed and direction. Also, blocks broadcast the actual setting of entry or exit turnouts if there are any. A layout control tool needs to be able to set the turnouts as part of setting a route across several blocks.

\section{Block Element Concept}

A block is the fundamental element of the block signaling system. The following figure shows the possible configuration of a block  element.

%!\href{./Figures/Block-Element-Concept.png }{Analog-Block-Element-Concept-Control.png}

From a moving direction viewpoint, each block can be seen as having exactly one entry point and optional two exits, depending whether a turnout is par of the block or not. If there is a turn out it is typically the last subsection of the block element. A block that can be entered from both directions potentially has a turnout at each end. In theory, any possible layout can be realized with this block element. The layout configuration and block control software will thus center around this basic element.

\section{Turnouts}

At a maximum, there are two possible turnouts for a block element. A block can have no turnouts, a turnout when entered in forward direction, and a turnout when entered in backward direction. Each turnout position, \textbf{ahead} and \textbf{turn} leads to another block. While the dependency between neighboring blocks are statically configured without turnouts (called switches in the UK), these dependency will become dynamic as soon there is a turnout at the beginning or end of a block. If there is a turnout (or group of turnouts) at the beginning of the block, the status of the block determines if the signal of the block the turnout is pointing can show red or green while the signal from the other direction have to show red by default. The same is true for a turnout at the end respectively. What the signal can display depends always on the status of the block the turnout points to.

\section{Signals}

Signals, also called semaphores in the US, communicate to the train engineer what is ahead. There are two kind of signals, the main signal at the end of a section and a distant signal associated with the main signal. The distant signals is placed in breaking distance seen from the associated main signal. 

Besides the usual red for \"stop\" and green for \"full steam ahead\" there can be a additional yellow light for go ahead at lower speed indicating there will be a turnout pointing to a siding.

\section{\textit{ABS and APB}}

With the basic elements  block, subsection, turnouts and signals in place, there are several ways to implement a block control system. The first algorithms is the \textbf{Automated Block System} ( \textbf{\textit{ABS}} ). In an ABS system the route is divided into blocks, each of which can at least hold the train length. A signal at the beginning controls whether that block can be entered at what speed. For example, a halt signal means that the train speed to enter the block is zero. The condition will is also shown with the distant signal. A distant signal is the signal ahead of the main signal that indicated the state of the main signal in breaking distance for the trains. So, a distant signal indicating a "slow" condition shows the condition of main signal ahead, which is the condition to enter the block ahead. The distance between the distance signal and the main signal is called the approach indication. Another variant is to just have main signals, one for each block entry. The approach indication in this case is the length of the block , the signal is the main signal of the block before. This variant can often be found in American railways.

% !\href{./Figures/Line-of-Blocks.png }{Line-of-Blocks.png}

So far, the signals refer to the series of blocks in one direction. Multiple trains can follow each each other in one direction. If the route is to be used in both directions, signals need to be placed also in the other direction. Also the route needs to be reserved, such that two trains do not enter the route from both ends. When a train is moving from A to B, signals guarding the route B to A will all show a halt condition and are cleared on a block by block fashion according to the movement of the train from block to block. To avoid that two trains enter a line from both directions and then have to stop in the middle, a kind of reservation system is needed to manage the current route direction. This system is called \textbf{Absolute Permissive Blocks} ( \textbf{\textit{APB}} ).

Both ABS and APB still have a fixed block length that accommodates the largest train length. But consider longer and shorter trains. Consider their speed. One approach go further traffic density would be to build a kind of moving block around the trains with a safety distance length at the rear and a breaking distance length at the front. A train following a train ahead is at least the own breaking distance and the safety distance f the train ahead away from that train. Such a system, called \textbf{Communication Based Train Control} ( \textbf{\textit{CBTC}} ) requires to know the precise location and speed of each train. Also, there is a central control center that authorizes and monitors the movements of all trains. Signals are no longer necessary.

// ??? \textbf{note} this is the general concept. Can we come up with a scheme that uses less blocks ? On a train station for example there are many turnouts on entry and exit and individual lanes in between them. Perhaps the turnouts can be combined to one block at the price of blocking the whole field when a train is entering or leaving ...

\section{Block Control Algorithms}

/// ??? \textbf{note} fill in ...

\subsection{Single Direction Line}

Imagine a single direction line with several blocks. The control element is the speed at which the train can enter to block ahead. If for example the block speed is zero, the train cannot enter this block. The following tasks need to be done periodically for each block.
\begin{itemize}
\item If the block is occupied, the block target speed to enter the block is zero. If not, the target speed is the target block speed currently specified.
\item If the target speed to enter the block is greater than zero, the next block is checked. If the target speed of the next block is lower than the target speed of this block, the target speed of this block is set to lower value of either the target speed of the next block plus an increment or the allowed maximum speed.
\item The entrance signal to this block is set according to the new target speed. If the target speed is lower than the previous one, the signal is set immediately. Otherwise, set the signal is set after a short delay time.
\end{itemize}

\subsection{Double Direction Line}

// ??? \textbf{note} fill in ...

\subsection{Absolute Permissive Blocks}

/// ??? \textbf{note} fill in ...


\section{Layering...}


\begin{center}
    \begin{tikzpicture}[scale=0.9, transform shape]
        
        \draw[help lines, gray!50, dashed] (0,0) grid(16,12);
    
        \node[  tsRoundedRectangle, 
                minimum width=10cm,
                minimum height=1.5cm,
                text width=3cm,
                text centered,
                fill=red!50] (orders) at (6,10) {Orders};
    
        \node[  tsRoundedRectangle, 
                minimum width=10cm,
                minimum height=1.5cm,
                text width=3cm,
                text centered,
                fill=red!50] (signaling) at (6,7) {Signaling};
    
        \node[  tsRoundedRectangle, 
                minimum width=10cm,
                minimum height=1.5cm,
                text width=3cm,
                text centered,
                fill=red!50] (blocks) at (6,4) {Blocks};
    
        \node[  tsRoundedRectangle,  
                minimum width=4cm,
                minimum height=1.5cm,
                text width=3cm,
                text centered,
                fill=yellow!50] (sensors) at (3,1) {Sensors};
    
        \node[  tsRoundedRectangle, 
                minimum width=4cm,
                minimum height=1.5cm,
                text centered,
                text width=3cm, fill=yellow!50] (actors) at (9,1) {Actors};
    
        \draw[<->, ultra thick, line cap=round] (orders.south) -- (signaling.north);
        \draw[<->, ultra thick, line cap=round] (signaling.south) -- (blocks.north);
        \draw[->, ultra thick, line cap=round] (sensors.north) -- ($(blocks.south) - (3,0)$);
        \draw[<-, ultra thick, line cap=round] (actors.north) -- ($(blocks.south) + (3,0)$);

        \draw[decorate,decoration={brace,amplitude=8pt,mirror,mirror}, line width=2pt] (12,0) -- (12,5);
        \node at (14, 2.5) {\parbox{2cm}{\textbf{Layout Elements}}};

        \draw[decorate,decoration={brace,amplitude=8pt,mirror,mirror}, line width=2pt] (12,6) -- (12,8);
        \node at (14, 7) {\parbox{2cm}{\textbf{Safety Rules}}};

        \draw[decorate,decoration={brace,amplitude=8pt,mirror,mirror}, line width=2pt] (12,9) -- (12,11);
        \node at (14, 9) {\parbox{2cm}{\textbf{Route Planning}}};
    
    \end{tikzpicture}
\end{center}

\section{Summary}

This chapter gave a brief overview on block signaling control concepts and the key algorithms to implement. Our block signaling system will implement single direction ABS, single line ABS and APB using signals on the line. The layout will be divided into blocks with subsections. The length of the block accommodates the largest train length and subsections are granularity of knowing where the trains actually is. Each block is managed by a block controller node, which manages the train hosted, associated signal and turnouts for the block. It also manages communicates the state to the other block controls and the central train authorization system. ( CTC ).

