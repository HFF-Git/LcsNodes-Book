\chapter{Introduction}

Model railroading. A fascinating hobby with many different facets. While some hobbyist would just like to watch trains running, others dive deeper into parts of their hobby. Some build a realistic scenery and model a certain time era with realistic operations. Others build locos and rolling equipment from scratch. Yet others enjoy the basic benchwork building, electrical aspects of wiring and control. They all have in common that they truly enjoy their hobby.

This little book is about the hardware and software of a layout control system for managing a model railroad layout. Controlling a layout is as old as the hobby itself. I remember my first model railroad. A small circle with one turnout, a little steam engine and three cars. Everything was reachable by hand, a single transformer supplied the current to the locomotive. As more turnouts were added, the arm was not long enough any more, simple switches, electrical turnouts and some control wires came to the rescue. Over time one locomotive did not stay alone, others joined. Unfortunately, being analog engines, they could only be controlled by electric current to the track. The layout was thus divided into electrical sections. And so on and so on. Before you know it, quite some cabling and simple electrical gear was necessary.

Nearly four decades ago, locomotives, turnouts, signals and other devices on the layout became digital. With growing sophistication, miniaturization and the requirement to model operations closer and closer to the real railroad, layout control became a hobby in itself. Today, locomotives are running computers on wheels far more capable than computers that used to fill entire rooms. Not to mention the pricing. Turnout control and track occupancy detection all fed into a digital control system, allowing for very realistic operations.

The demands for a layout control system can be divided into three areas. The first area is of course {\bf running} locomotives. This is what it should be all about, right? Many locomotives need to be controlled simultaneously. Also, locomotives need to be grouped into consists for large trains, such as for example a long freight train with four diesel engines and fifty boxcars. Next are the two areas {\bf observe} and {\bf act}. Track occupancy detection is a key requirement for running multiple locomotives and knowing where they are. But also, knowing which way a turnout is set, the current consumption of a track section are good examples for layout observation. Following observation is to act on the information gathered. Setting turnouts and signals or enabling a track section are good examples for acting on an observation.

Running, observing an acting requires some form of {\bf configurations} and {\bf operations} What used to be a single transformer, some cabling and switches has turned into computer controlled layout with many devices and one or more bus systems. Sophisticated layouts need a way to configure the locomotives, devices and manage operations of layouts. Enter the world of digital control and computers.

After several decades, there is today a rich set of product offerings and standards available. There are many vendors offering hardware and software components as well as entire systems. Unfortunately they are often not compatible with each other. Furthermore, engaged open software communities took on to build do it yourself systems more or less compatible with vendors in one or the other way. There is a lively community of hardware and software designers building hardware and software layout control systems more or less from scratch or combined using existing industry products.

\section{Elements of a Layout Control System}

Before diving into concept and implementation details, let's first outline what is needed and what the resulting key requirements are. Above all, our layout control system should be capable to simultaneously run locomotives and manage all devices, such as turnouts and signals, on the layout. The system should be easy to expand as new ideas and requirements surface that need to be integrated without major incompatibilities to what was already built.

Having said that, we would need at least a {\bf base station}. This central component is the heart of most systems. A base station needs to be able to manage the running locomotives and to produce the DCC signals for the track where the running locomotive is. There are two main DCC signals to generate. One for the main track or track sections and one for the programming track. This is the track where a locomotive decoder can be configured. A base station could also be the place to keep a dictionary of all known locomotives and their characteristics. In addition to interfaces to issues commands for the running locomotives, there also need to be a way to configure the rolling stock.

Complementing the base station is the {\bf booster} or {\bf block controller} component that produce the electrical current for a track section. The booster should also monitor the current consumption to detect electrical shortages. Boosters comes in several ranges from providing the current for the smaller model scales as well as the larger model scales which can draw quite a few amps. There could be many boosters, one for each track section. The base station provides the signals for all of them.

The {\bf cab handheld} is the controlling device for a locomotive. Once a session is established, the control knobs and buttons are used to run the locomotive. Depending on the engine model, one could imagine a range of handhelds from rather simple handhelds just offering a speed dial and a few buttons up to a sophisticated handheld that mimics for example a diesel engine cab throttle stand.

With these three elements in place and a communication method between them, we are in business to run engines. Let's look at the communication method. Between the components, called nodes, there needs to be a {\bf communication bus} that transmits the commands between them. While the bus technology itself is not necessarily fixed, the messaging model implemented on top is. The bus itself has no master, any node can communicate with any other node by broadcasting a message, observed by all other nodes. Events that are broadcasted between the nodes play a central role. Any node can produce events, any node can consume events. Base station, boosters and handhelds are just nodes on this bus.

But layouts still need more. There are {\bf signals}, {\bf turnouts} and {\bf track detectors} as well as {\bf LEDs}, {\bf switches}, {\bf buttons} and a whole lot more things to imagine. They all need to be connected to the common messaging bus. The layout control system needs to provide not only the hardware interfaces and core firmware for the various device types to connect, it needs to also provide a great flexibility to configure the interaction between them. Pushing for example a button on a control field should result in a turnout being set, or even a set of turnouts to guide a train through a freight-yard and so on.

Especially on larger layouts, {\bf configuration} becomes quite an undertaking. The {\bf configuration model} should therefore be easy and intuitive to understand. The elements to configure should all follow the same operation principles and be extensible for specific functions. A computer is required for configuration. Once configured however, the computer is not required for operations. The capacity, i.e. the number of locomotives, signals, turnouts and other devices managed should be in the thousands.

Configuration as well as operations should be possible through sending the defined messages as well as a simple ASCII commands send to the base station which in turn generates the messages to broadcast via the common bus. A computer with a graphical UI would connect via the USB serial interface using the text commands.

\section{Standards, Components and Compatibility}

The DCC family of standards is the overall guiding standard. The layout system assumes the usage of DCC locomotive decoder equipped running gear and DCC stationary decoder accessories. Beyond this set of standards, it is not a requirement to be compatible with other model railroad electronic products and communication protocols. This does however not preclude gateways to interact in one form or another with such systems. Am example is to connect to a LocoNet system via a gateway node. Right now, this is not in scope for our first layout system.

All of the project should be well documented. One part of documentation is this book, the other part is the thoroughly commented LCS core library and all software components built on top. Each lesson learned, each decision taken, each tradeoff made is noted, and should help to understand the design approach taken. Imagine a fast forward of a couple of years. Without proper documentation it will be hard to remember how the whole system works and how it can be maintained and enhanced.

With respect to the components used, it uses as much as possible off the shelf electronic parts, such as readily available microcontrollers and their software stack as well as electronic parts in SMD and non-SMD form, for building parts of the system. The concepts should not restrict the development to build it all from scratch. It should however also be possible to use more integrated elements, such as a controller board and perhaps some matching shields, to also build a hardware module.

\section{This Book}

This book will describe my version of a layout control system with hardware and software designed from the ground up. The big question is why build one yourself. Why yet another one? There is after all no shortage on such systems readily available. And there are great communities out there already underway. The key reason for doing it yourself is that it is simply fun and you learn a lot about standards, electronics and programming by building a system that you truly understanding from the ground up. To say it with the words of Richard Feynman

\begin{quotation}
    \textit{ "What I cannot create, I do not understand. -- Richard Feynman"}
\end{quotation}

Although it takes certainly longer to build such a system from the ground up, you still get to play with the railroad eventually. And even after years, you will have a lay  out control system properly documented and easy to support and enhance further. Not convinced? Well, at least this book should be interesting and give some ideas and references how to go after building such a system.

\section{Parts and Chapters}

The book is organized into several parts and chapters. The first chapters describe the underlying concepts of the layout control system. Hardware modules, nodes, ports and events and their interaction are outlined. Next, the set of message that are transmitted between the components and the message protocol flow illustrate how the whole system interacts. With the concepts in place, the software library available to the node firmware programmer is explained along with example code snippets. After this section, we all have a good idea how the system configuration and operation works. The section is rounded up with a set of concrete programming examples.

Perhaps the most important part of a layout control system is the management of locomotives and track power. After all, we want to run engines and play. Our system is using the DCC standard for running locomotives and consequently DCC signals need to be generated for configuring and operating an engine. A base station module will manage the locomotive sessions, generating the respective DCC packets to transmit to the track. Layouts may consist of a number of track sections for which a hardware module is needed to manage the track power and monitor the power consumption. Finally, decoders can communicate back and track power modules need to be able to detect this communication. Two chapters will describe these two parts in great detail.

The next big part of the book starts with the hardware design of modules. First the overall outline of a hardware module and our approach to module design is discussed. Building a hardware module will rest on common building blocks such as a CAN bus interface, a microcontroller core, H-Bridges for DCC track signal generation and so on. Using a modular approach the section will describe the building blocks developed so far. It is the idea to combine them for the purpose of the hardware module.

With the concepts, the messages and protocol, the software library and the hardware building blocks in place, we are ready to actually build the necessary hardware modules. The most important module is the base station. Next are boosters, block controllers, handhelds, sensor and actor modules, and so on. Finally, there are also utility components such as monitoring the DCC packets on the track, that are described in the later chapters. Each major module is devoted a chapter that describes the hardware building blocks used, additional hardware perhaps needed, and the firmware developed on top of the core library specifically for the module. Finally, there are several appendices with reference information and further links and other information.

\section{A final note}

A final note. "Truly from the ground up" does not mean to really build it all yourself. As said, there are standards to follow and not every piece of hardware needs to be built from individual parts. There are many DCC decoders available for locomotives, let's not overdo it and just use them. There are also quite powerful controller boards along with great software libraries for the micro controllers, such as the CAN bus library for the AtMega Controller family, already available. There is no need to dive into all these details.

The design allows for building your own hardware just using of the shelf electronic components or start a little more integrated by using a controller board and other breakout boards. The book will however describe modules from the ground up and not use controller boards or shields. This way the principles are easier to see. The appendix section provides further information and links on how to build a system with some of the shelf parts instead of building it all yourself. With the concepts and software explained, it should not be a big issue to build your own mix of hardware and software.

I have added most of the source files in the appendix for direct reference. They can also be found also on GitHub. ( Note: still to do... ) Every building block schematic shown was used and tested in one component or another. However, sometimes the book may not exactly match the material found on the web or be slightly different until the next revision is completed. Still, looking at portions of the source in the text explain quite well what it will do. As said, it is the documentation that hopefully in a couple of years from now still tells you what was done so you can adapt and build upon it. And troubleshoot.

The book hopefully also helps anybody new to the whole subject with good background and starting pointers to build such a system. I also have looked at other peoples great work, which helped a lot. What I however also found is that often there are rather few comments or explanations in the source and you have to partially reverse engineer what was actually build for understanding how things work. For those who simply want to use an end product, just fine. There is nothing wrong with this approach. For those who want to truly understand, it offers nevertheless little help. I hope to close some of these gaps with a well documented layout system and its inner workings.

In the end, as with any hobby, the journey is the goal. The reward in this undertaking is to learn about the digital control of model railroads from running a simple engine to a highly automated layout with one set of software and easy to build and use hardware components. Furthermore, it is to learn about how to build a track signaling system that manages analog and digital engines at the same time. So, enjoy.



