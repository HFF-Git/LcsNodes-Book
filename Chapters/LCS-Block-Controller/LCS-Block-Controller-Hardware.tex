\chapter{The Block Controller Node Hardware}

So far, we covered a lot of ground. The Layout Control System rests on the concept of a node with ports and a common bus for nodes to communicate with each other. There are specialized nodes, one of them being the base station responsible for managing the locomotive sessions and generating the DCC signals. The base station is as the name suggest a central piece in a layout. The base station node shown in the previous chapter featured two track outputs, MAIN and PROG. The MAIN track is what powers the layout. However, the base station power module unit allowed for about 2.5 Amps and hence only smaller layouts will just do fine with one base station. For larger layouts, help to power several track sections is needed.

A common approach is to divide the layout in several sections or blocks and power them individually with a booster or block controller. Such a booster or block controller would be very similar to a base station in that it provides control about the track, current consumption limits and so on. In contrast to a base station a booster or block controller would not produce its own DCC signal or manage a locomotive session but rather use the signal distributed via the LCS bus. Being an LCS node, the block controller will offer a rich set of attributes to manage and query the state and configuration. Again, wherever the functionality is the same that can be found on the base station, there will be the same attributes and functions.

But there is more to it. For a safe and automated operation, dividing the layout on to track blocks with block sections, and block signaling concepts are necessary. A train movement is always a planned movement and hence the track route needed for the movement is authorized from a central operations place. The chapter on block signaling control provided a high level overview on the block signaling concepts. This chapter will provide a big step into automating a layout by introducing the block controller.

Conceptually, a booster and a block controller have very much in common. They both handle a section or block, they both need power modules to generate the track power, they both need to offer basic sensor and control capabilities to manage a track section or block. Our concept will just unify these two modules types and we will call them block controller from now on. A block controller is a LCS node that manages a track. There are versions with one channel, i.e. one block, and two channels. Support for track occupancy detection, an essential feature for block control, is optional since simple boosters would not need it if they just manage one block with no sections. The text will just use the term block controller and only mention boosters explicitly when needed. Let's get started with the overall requirements.

\section{Requirements}

Traditionally, block control in a layout system was implemented with central control electronics and/or relays. This introduced not only a significant amount of cabling but also a high complexity in building a block control system. A centralized place for all the parts of a block system also results in long cable connections running in parallel which act like antennas resulting in face signals. As a general rule, cable connections on a model railroad should be as short as possible. With a decentralized concept where each block or a small number of blocks are managed by a dedicated controller, the lines to tracks, detectors and signals can be much shorter. There may be still a central base station, but the cables to track, switches and signals are just connected to local booster units located next to them. Over the years, the rise of digital control helped to simplify the configuration and building but the central approach still dominates most designs along with the high cabling effort. A digital system, for example, will still need the track occupancy network, perhaps RailCom detectors and a network of boosters to control the layout sections.

As discussed before a large part of the hardware building blocks to run analog and digital equipment in a block system are identical. Both need the same track sections with occupancy detectors and a H-bride as a power module. All the rest can be done in software. If the engine-ID indicate an analog engine the H-bridge will power the track via PWM signals, in case of a digital engine the H-bridge power the track with DCC signals. So, running trains with analog and digital behind each other is just a question of software and without need for additional hardware. You just can’t mix analog and digital on the same block.

A layout is divided into blocks and each block has several sections. For each block and each section in that block a way needs to be found to feed either analog PWM style signals or digital DDC or other protocol based signals along with the safety mechanisms for engines of either type moving from block to block. A key requirement for our block controller is to accommodate this mixed mode running engines. A block is the unit of control on a layout with block signaling. For configuration purposes it has a unique ID, the block Id. Since a layout is fixed once built, the linkages between blocks is too. A block has as the most important attribute the block entry speed. For example, a block occupied has an entry speed of zero. That is an engine following needs to stop at the signal guarding the entry to the occupied block. Block speed can of course vary. For a free block it is the configured maximum block speed, for an occupied block it is zero, and otherwise anything in between depending on the state of the following block and the type and speed of the train.

Blocks can be entered from only one or both directions. While this requirement does not change the basic mechanism of block section occupancy detection, it will change the algorithms implemented. The block algorithm for bi-directional mode routes needs to put at least two stop blocks ahead of them between two trains entering a route to avoid a head on collision. Still, for a bidirectional line, the route direction needs to be set to avoid two trains entering that route in the first place. Our controller needs to be able to be part of a route configuration process. When talking about routes the key requirements is that each block and possible a turnout in the block can be managed centrally from a control panel. All blocks need to offer stair state on demand. The signals along the route are entirely controlled by the state, i.e entry speed, of the block.They are just indicators. Modern block signaling systems do not even have signals, but most model railroader will have signals just to make the layout more interesting. But imagine a staging yard where the block control algorithms will still be there but no signals are there.

Both analog and digital block modes need a way to know where the train actually is within the block and a way to manage the speed change. For example, when the next block is having an entry speed of zero, the train needs to be decelerated to that speed, i.e. it needs to halt at the end of the block. Whether analog or digital, it will translate to either sending digital commands to the engine, or to reducing the track voltage for analog systems. A key requirement is therefore either knowing exactly where the train is within the block or having a way to measure the actual speed to change it accordingly. This topic is explained further in the firmware section of this chapter.

Blocks can have an optional turnout on the entry and the exit. In other words, there are up to two entries to a block and up to two exits. This basic element forms the building block for a layout. The actual setting of the turnout and track occupancy influences the route checking and need also to be broadcasted to all interested parties. The requirement is that the neighbor blocks and as well as any control panel need to receive these events and act accordingly. During power up or restart after a power failure, the setting of the turnout need to be recreated. The requirement is either that the actual setting can be queried or that the last setting is stored in a non-volatile memory so that after a restart the turnout can just be set. However, the safest way is to have sensors for detecting the actual turnout setting.

For digital operations, there is also the requirement for supporting RailCom. A block controller extension should have the optional capability to include a RailCom detector. This detector is also used to identify the actual engine occupying the block. For analog operations the block controller can only detect that there is an engine. Other ways of identifying the engine are needed. It is not a key requirement per se that analog and digital operations are possible simultaneously. But as described in the analog operations chapter there are good reasons to support both modes in one layout. If implemented, the overall layout control needs to manage the type of engine occupying a block and ensure that the next block to enter has the same mode. More on this in the firmware section of this chapter.

\section{Overall hardware module design}

For the design of the hardware module, there are two basic ways. As said, one way is to implement a central approach to the management of all the blocks, the other is a decentralized approach where each block is managed separately. Our block control system implements the latter. It is a decentralized system where a block controller manages two, perhaps four blocks. There is no real limit, but the more blocks are managed by one block controller, the more cabling is required, up to having one block controller managing all blocks, and thus we are back to a central approach. The requirements for a decentralized approach are that all components required to manage a block are bundled in the block controller and perhaps extension board.

Each \textbf{block controller board} will have a main controller based board with the power module circuitry and RailCom detectors. This board alone will also cover all features needed for a booster node. In addition, the\textbf{ block controller extension board} will cover the needs for occupancy detection, turnout and signal control. Depending on the hardware types, there are different turnouts and signal drivers. Again, we could design a separate extension board for the track occupancy and another one for the turnout and signal control. Both can be connected to the block controller main board.

While all this may sound like a hardware overkill, the occupancy detectors the turnout and signals drivers need to be in place anyhow. So are the boosters. There are only the number of power modules, one per block, instead of few boosters along the layout required. But having a power module for each bock gives us a good way to manage analog and digital blocks in one layout. And the prices for a H-Bridge chip that controls two blocks is quite reasonable so far.

\section{Summary}

we favor the block controller node approach

next chapters will present HW and then SW
