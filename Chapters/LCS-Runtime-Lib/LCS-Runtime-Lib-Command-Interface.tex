%-------------------------------------------------------------------------------------------------------
\chapter{RtLib Command Interface}

???  explain the general concept ...

The primary communication method of the layout control system are LCS messages sent via the bus. In addition, each module that offers an USB connector or the serial I/O connector, implements also the serial command console interface. The interface is intended for testing and tracing purposes. LCS console commands are entered through the hardware module serial interface. 

Perhaps the most important command is the help command, which lists all available command and  their basic syntax.

\lstset{style=codesnippetstyle}
\begin{lstlisting}
<!?>
<#?>
\end{lstlisting}

Any command not recognized is passed to a command line handler....

<lcs-command-char [ arguments ] >


will be passed to the registered command call back function, if there is one registered. The following summary shows the available LCS serial commands. The appendix contains a detailed description of of the commands implemented by the LCS library.

\section{Configuration Mode Commands}

The configuration mode commands will place a node into either operations or configuration mode.

%|Command | Arguments | Operation |
%|:----|:-----|:-----|
%|!c | | enter node configuration mode |
%|!o | | enter node operations mode |

\section{Event Commands}

Event commands work with the event map. They add and remove an event, search the map for an event/port pair, or locally send an event to the node itself to test the event handling and so on.

%%|:----|:-----|:-----|
%|!a | eventId&nbsp; \[portId\] | add an event to the eventMap. If the portId is omitted, all eventMap %entries with a matching eventId will be removed.  |
%|!r | eventId&nbsp;\[portId\]| remove an event the eventMap. If the portId is omitted, all eventMap entries with a matching eventId will be removed. |
%|!f | eventId&nbsp; \[ portId \] | search an event in the eventMap. |
%|!e | mode&nbsp;nodeId&nbsp;eventId&nbsp;\[arg\] | simulate sending an event ( mode: 0 - ON, 1 - OFF, 2 - EVT ) |

\section{Node Map and Attributes Commands}

The node map and attribute map will examine and modify these maps.

%%|:----|:-----|:-----|
%|!n| item \[arg1\[arg2\]\]| lists a node or port map item.|
%|!N| item \[arg1\[arg2\]\]| sets a node or port map item.|
%|!v | attrId \[mode\] | list a global attribute. |
%|!V | attrId mode val | sets a a global attribute.|

\section{Send a raw Message}

For testing the message send mechanism, a command is available to send a raw data packet via the LCS bus.

%|Command | Arguments | Operation |
%|:----|:-----|:-----|
%|!B | byte1 [ byte2 .. byte8 ] | send a raw LCS message |

\section{List node status}

The "s" command will list a great detail on the node data. When debugging a node problem, this is perhaps the most useful command to see what is store locally.

%|Command | Arguments | Operation |
%|:----|:-----|:-----|
%|!s | \[level\] | list status at detail level, default is summary. ( 1 - ConfigDesc, 2 - NodeMap, 3 - %PortMap, 4 - EventMap, 6 - NVM Area, 7 - MEM Area ) |

\section{Driver commands}



What about the "xxx" commands? Well, they are used issue commands to the hardware drivers. We have not talked about them so far. This topic is presented when we know more about how the hardware is structured. Stay tuned.

\section{LCS message text format}

Just like the LCS core library accepts simple ASCII command strings, the LCS messages can also be transmitted as an ASCII text line. This is very useful for building communication gateways that transmit the message via another medium, such as an ethernet channel. There is a simple scheme for the ASCII representation of the message:

%```
%<$ data-byte-1 ... data-byte-n>
%

The message is enclosed in the "<" and ">" delimiters and the first character is the "xxx" sign. Up to 8 hexadecimal values written as "0xdd" follow, where "d" is a hexadecimal digit.

%%```cpp
%int lcsMsgToStr( char *msgStr, uint8_t *dataBuf, uint8_t dataBufLen );

%int strToLcsMsg( uint8_t *dataBuf, uint8_t dataBufLen, char *msgStr );
%```

Note: to be implemented. Perhaps to simple library routines to create an ASCII version of a LCS message and convert an ASCII string to an LCS message.

\section{Summary}

The command line interface provides a way to interact with a node at the command line level. This is very useful for initial testing new hardware and software debugging. All that is needed is a USB interface and a computer. As we will see in the main controller chapter, a USB or serial interface is also necessary for downloading new firmware to the boards. Besides that, this interface is normally not used during regular operations.