%-------------------------------------------------------------------------------------------------------
%-------------------------------------------------------------------------------------------------------
\chapter{RtLib Call Interface}

??? this chapter needs to be reworked for new library call interface....

The LCS runtime library is the foundation for any module firmware written. The library presents to the firmware programer a set of routines to configure, manage the LCS node and use the LCS functions, such as sending a message. This chapter will present the key functions used. We will look at library initialization, obtaining node information, controlling a node aspect, reacting to an event and sending message to other nodes.
Refer to the appendix for a complete set of available LCS runtime functions.

\section{Library initialization}

The LCS runtime is initialized with the \textbf{init} routine. After successful runtime initialization, the firmware programmer can perform the registration of the callback functions needed, as well as doing other node specific initialization steps. This also includes the setup of the particular hardware. The subject of hardware setup will be discussed in a later chapter, "controller dependent code". 

While there are many library functions to call, the only way for the library to communicate back to the module firmware when a message is received are the callbacks registered for. Callbacks will be described in the next chapter. A key task therefore is to register call back functions for all events and messages the node is interested. The following code fragment illustrates the basic library initialization.

\lstset{language=c++, style=codesnippetstyle}
\begin{lstlisting}
   
    code snippet here \dots

\end{lstlisting}


The final library call is a call to \textbf{run}. The run function processes the incoming LCS messages, manages the port event handling, reacts to console commands and finally invokes user defined callback functions. Being a loop, it will not return to the caller, but rather invoke the registered callback functions to interact with the node specific code. Before talking about the callback routines, let's have a look at the local functions available to the  programmer to call functions in the core library.

\section{Obtaining node information}

Obtaining node or port information is an interface to query basic information about the node or port. A portID or \texttt{NIL\_PORT\_ID} will refer to the node, any other portID to a specific port on that node. The data is largely coming from the nodeMap and portMap data structures. The LCS library defines a set of data items that can be retrieved. 

The return result is stored in one or two 16-bit variables and is request item specific. The nodeInfo and nodeControl routines allow for local access, the \texttt{(QRY-NODE)} and \texttt{(REP-NODE)} messages allow for remote access. The following example shows how the number of configured ports is retrieved from the nodeMap.


\lstset{language=c++, style=codesnippetstyle}
\begin{lstlisting}
   
    code snippet here \dots
    
\end{lstlisting}

\section{Controlling a node aspect}

Very similar to how we retrieve node data, the nodeControl routine allows for setting node attribute. A node attribute does not necessarily mean that there is a data value associated with the attribute. For example, turning on the "ready" LED is a control item defined for the nodeControl routine. There is a detailed routine description in the appendix that contains the items that are defined. The following example turns on the ready LED on the module hardware.

\lstset{language=c++, style=codesnippetstyle}
\begin{lstlisting}
   
    code snippet here \dots
    
\end{lstlisting}

The example shows that a node item is not only used to read or write a data item. It can also be used to execute a defined command, such as turning on an LED. In addition to the predefined node items, there is room for user defined items. In order to use them, a callback function that handles these items needs to be registered. This concept allows for a very flexible scheme how to interact with a node.

\section{Controlling extension functions}

// ??? the extension and driver stuff....

\section{Reacting to events}


// ??? rather a callback topic ?


\section{Sending messages}

Sending a message represent a large part of the available library functions. For each message defined in the protocol, there is a dedicated convenience function call, which will take in the input arguments and assemble the message buffer accordingly. As an example, the following code fragment will broadcast the ON event for event "200".

\lstset{language=c++, style=codesnippetstyle}
\begin{lstlisting}
   
    code snippet here \dots
    
\end{lstlisting}

All message sending routines follow the above calling scheme. The data buffer is assembled and out we go. Transparent to the node specific firmware, each message starts with a predefined messages priority. If there is send timeout, the priority will be raised and the message is sent again. If there is a send timeout at the highest priority level, a send error is reported.

\section{Summary}

A key part of the runtime library is the setup and manipulation of node and port data. A small comprehensive function set was presented in this chapter. That is all there is to invoke the core library functions. There are a few more functions that will be described in the chapters that deal with their purpose. For the other direction of information flow, i.e. the core library sends information back to the firmware layer, callback functions are used, presented in the following chapter.
